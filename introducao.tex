\part{Introdução}
	\chapter{Surfactantes}
	
	Surfactantes são moléculas que possuem regiões com polaridades diferentes, são anfifílicas. Uma das regiões interage com com o solvente, e portanto é chamada de liofílica, e a outra interage mal, chamada de liofóbica. Quando o solvente é água, essas regiões são chamadas então hidrofílicas e hidrofóbicas, respectivamente. Essas diferenças de afinidade resultam na orientação das moléculas de surfactante de acordo com as moléculas ao redor, de maneira a maximizar a quantidade de interações favoráveis.
	
	Em solução aquosa, as moléculas de surfactante se orientam na interface com o ar, com a região apolar direcionada para o ar, e a região polar direcionada para a água. Esse efeito está relacionado à maior favorabilidade das interações polares com a água. É necessário, entretanto, considerar também as moléculas de solvente. Para maximizar as ligações de hidrogênio, as moléculas de água devem se organizar ao redor da cadeia hidrofóbica, resultando numa perda alta de entropia. Caso essas moléculas de água sejam liberadas, por exemplo, orientando a região apolar para o ar, o ganho entrópico das moléculas de água se sobrepõe à perda entrópica das moléculas de surfactante. Esse efeito, conhecido como efeito hidrofóbico, é um força motriz bastante importante da química coloidal.
	
	À medida que mais surfactante é adicionado, mais moléculas se concentram na superfície, até uma concentração limite. Nessa concentração, as moléculas começam a formar estruturas de autoassociação, como, por exemplo, micelas. Da mesma forma que no caso das moléculas de surfactante na superfície, a perda entrópica da formação de micelas é compensada pelo ganho entrópico das moléculas de água. 
	
	% todo: verificar bem os casos onde isso acontece.
	
		\section{Polaridade}
		\section{Parâmetro de empacotamento}
		\section{Mesofases}
		\section{Aditivos}
	\chapter{Micelas gigantes}
		\section{Crescimento de micelas}
		\section{Termodinâmica de micelas}
		\section{Modelos de comportamento reológico}
		\section{Perfis de viscosidade}
	\chapter{Inspirações para o projeto}
		\chapter{Estudos de Hoffmann sobre micelas e lamelas}
		\chapter{Estudos de Pedersen sobre cinética}
	\chapter{Objetivos}
