% todo: pensar em colocar uma parte falando sobre o método dos mínimos quadrados
\part{Teoria}
	\chapter{Reologia}
		\section{Fundamentos}
			\subsection{Fluídos Newtonianos}
			\subsection{Sólidos Hookeanos}
			\subsection{Fluidos viscoelásticos}
		\section{Reologia oscilatória}
			\subsection{Aquisição de dados}
			\subsection{Modelo de Maxwell}
			\subsection{Modelos mais complexos}
		\section{Curvas de Fluxo}
			\subsection{Modelos de curvas de fluxo}
			% todo: colocar a seção dos apêndices onde eu descrevo as curvas aqui, e depois só referenciar o apêndice para a implementação técnica dos modelos
	\chapter{Calorimetria de titulação isotérmica}
		\section{Fundamentos}
			\subsection{Aquisição de dados}
		\section{Calorimetria de micelas esféricas}
		\section{Calorimetria de micelas gigantes}
		\section{Termodinâmica de micelização}
	\chapter{SAXS}
		\section{Fundamentos}
		\section{Modelagem}
			\subsection{Esferas}
			\subsection{Micelas esféricas}
			\subsection{Micelas gigantes}
			\subsection{Visualização dos parâmetros}
			\subsection{Indexação de picos}
	\chapter{Fluorescência}
		\section{Fundamentos}
			\subsection{Diagramas}
			\subsection{Rendimento quântico}
			\subsubsection{Lei de X (não importa onde incide para fluorescência)}
	\chapter{Coloides}
	% todo: falar sobre a constante de Hamaker e outras coisas importantes para a discussão
		\section{Atração coloidal}
		% Falar sobre como as atrações de vdW em coloides são aditivas, e sobre o problema enfrentado.
		% Falar sobre os estudos do físico para tentar contornar esse problema.
		% Falar sobre DLVO
		\section{Constante de Hamaker}
		% Colocar a equação e falar como que se obtêm os valores
	\chapter{Micelas gigantes}
		\label{chap:micelas_gigantes}
		\section{Crescimento de micelas}
		% Falar sobre o diagrama de fases de micelas
		% Procurar um pouco, na tese da Danila e em teses anteriores do grupo, sobre o que falar a mais de micelas
		% todo: completar esta seção
		As micelas crescem.
		% Falar sobre os tempos de relaxação vistos por Hoffmann por birrefringência elétrica
		\section{Termodinâmica de micelas}
		% Falar sobre a energia das pontas. Pesquisar sobre o que falar.
		% Fazer uma breve introdução sobre ITC e o que se observa.
		% Falar sobre o ITC e como ele consegue enxergar os processos de micelização
		% todo: fazer esta parte
		As micelas tem pontas com energia superior.
		\section{Comportamento reológico}
		% Falar sobre como identificar visualmente micelas gigantes, como o recoil
		% Falar sobre as várias estruturas que elas possuem, e como a concentração de salicilato afeta as estruturas
		% Falar sobre as G', G'' e o modelo de Maxwell.
		% Falar sobre o tempo de relaxação micelar.
		% Falar sobre os sticky contacts que o Hoffmann tanto gosta.
		\section{Efeito do solvente}
		\label{sec:efeito_solvente}
		% Falar mais profundamente sobre os estudos do Hoffmann
		% Falar sobre a hidrofilicidade da superfície micelar, o t3 de relaxação e a relação com os perfis de viscosidade.
		
		
%	\chapter{Análise Multivariada}
%		\section{Técnicas de classificação}
%			\subsection{Normalização dos dados}
%			\subsection{PCA}
%			\subsection{HCA}
%		\section{Técnicas de regressão}
%			\subsection{Regressão Multivariada}
%			\subsection{PCR}
%			\subsection{PLS}