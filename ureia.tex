\chapter{Efeito da ureia}
	
\section{Motivação}
A ureia demonstrou um comportamento que divergiu bastante dos outros aditivos. Por esse motivo, ela será estudada um pouco mais profundamente. Foi estudado principalmente o efeito da ureia em soluções de CTAB, TTAB e DTAB, com concentrações diferentes de ureia e surfactante. O efeito do salicilato não foi muito estudado de modo a simplificar o sistema.

% TODO: Verificar se o nome "esbranquiçada" realmente é bom o suficiente.

Em concentrações de ureia maiores que 35\%, ocorre a formação de uma fase esbranquiçada, viscosa, a temperatura ambiente. A amostra se torna totalmente transparente e pouco viscosa quando aquecida. Essa temperatura de transição foi estudada utilizando-se calorimetria diferencial de varredura (DSC). A identidade da mesofase, por sua vez, foi estudada a partir de espalhamento de raios-X em baixos ângulos (SAXS). A estruturação das soluções em temperaturas acima da temperatura de transição foi estudada por espalhamento dinâmico de luz (DLS). Também foram estudadas a reologia da fase esbranquiçada e o calor de interação entre surfactante e ureia por calorimetria de titulação isotérmica.

\section{Calorimetria diferencial de varredura (DSC)}

	Foram preparadas soluções de ureia, em várias concentrações, com três surfactantes (CTAB, TTAB e DTAB), em três concentrações. Os termogramas resultantes foram organizados em figuras de modo a facilitar comparações. A tabela \ref{tab:refs_DSC} lista as comparações realizadas e em quais figuras estão.
    
    \begin{table}[H]
        \IBGEtab{%
            \caption{Comparações de termogramas de amostras de surfactante (concentração em \mM)e ureia (\% m/m), e suas respectivas figuras}
            \label{tab:refs_DSC}
            }%
            {%
            \begin{tabular}{l c c}
                \toprule
    			%\centering
				Surf. e Conc. em \mM             & \% Ureia		 & Figura 			\\
    			\midrule
				CTAB 100	                     & 38---45		 & \ref{fig:DSC_CTAB_UR38-45}	\\
				CTAB 100, 200, 300	             & 45, 40	     & \ref{fig:DSC_CTAB_UR40-45}	\\
				TTAB 100, 200, 300	             & 45, 40	     & \ref{fig:DSC_TTAB_UR_40-45}	\\
				DTAB 100, 200, 300	             & 45, 40	     & \ref{fig:DSC_DTAB_UR_40-45}	\\
				% CTAB, TTAB, DTAB 100	         & 45	         & \ref{fig:DSC_Surf_100mm_45p}	\\
				% CTAB, TTAB, DTAB 200	         & 45	         & \ref{fig:DSC_Surf_200mm_45p}	\\
				% CTAB, TTAB, DTAB 300	         & 45	         & \ref{fig:DSC_Surf_300mm_45p}	\\
    			\midrule
				CTAB 100 NaSal 60	             & 35, 40, 45	 & \ref{fig:DSC_NaSal60}		\\
				CTAB 100 NaSal 100	             & 35, 40, 45	 & \ref{fig:DSC_NaSal100}	\\
				CTAB 100 NaSal 250	             & 35, 40, 45	 & \ref{fig:DSC_NaSal250}	\\
				% CTAB 100 NaSal 60, 100, 250    & 35 	         & \ref{fig:DSC_NaSal_Ur35}  \\
				% CTAB 100 NaSal 60, 100, 250    & 40	             & \ref{fig:DSC_NaSal_Ur40}  \\
				% CTAB 100 NaSal 60, 100, 250	 & 45	         & \ref{fig:DSC_NaSal_Ur45}  \\
    			\bottomrule
            \end{tabular}%
            }{}
    \end{table}
	
	% todo: cortar o número de figuras para diminuir repetição, adicionar outros gráficos mostrando como os parâmetros são afetados
	% todo: tirar CTAB 400 mM? Ele foje em muitos casos do esperado. Como explicar?
	
	\begin{figure}[H]
		\centering
		\includegraphics[width=0.75\textwidth]{./imagens/dsc/CTAB_porc_ur}
		\caption{Termogramas de soluções de CTAB 100 \mM{} em concentrações crescentes de ureia, de 38\% m/m a 45\% m/m}
		\label{fig:DSC_CTAB_UR38-45}
	\end{figure}
	
	\begin{figure}[H]
		\begin{subfigure}[t]{0.45\textwidth}
			\centering
			\includegraphics[width=\textwidth]{./imagens/dsc/CTAB_45p}
			\caption{45\% de ureia}
			\label{fig:DSC_CTAB_UR45}
		\end{subfigure} \qquad %
		\begin{subfigure}[t]{0.45\textwidth}
			\centering
			\includegraphics[width=\textwidth]{./imagens/dsc/CTAB_40p}
			\caption{40\% de ureia}
			\label{fig:DSC_CTAB_UR40}
		\end{subfigure}
		\caption{Termogramas de CTAB 100, 200 e 300 \mM{} em soluções em 45\% (\ref{fig:DSC_CTAB_UR45}) e 40\% (\ref{fig:DSC_CTAB_UR40})}
		\label{fig:DSC_CTAB_UR40-45}
	\end{figure}

	\begin{figure}[H]
		\centering
		\begin{subfigure}[t]{0.45\textwidth}
			\includegraphics[width=\textwidth]{./imagens/dsc/TTAB_45p}
			\caption{45\% de ureia}
			\label{fig:DSC_TTAB_UR45}
		\end{subfigure} \qquad %
		\begin{subfigure}[t]{0.45\textwidth}
			\includegraphics[width=\textwidth]{./imagens/dsc/TTAB_40p}
			\caption{40\% de ureia}
			\label{fig:DSC_TTAB_UR40}
		\end{subfigure}
		\caption{Termogramas de soluções de TTAB 100, 200 e 300 \mM{}, em 45\% (\ref{fig:DSC_TTAB_UR45}) e 40\% (\ref{fig:DSC_TTAB_UR40}) de ureia}
		\label{fig:DSC_TTAB_UR_40-45}
	\end{figure}
	
	\begin{figure}[H]
		\centering
		\begin{subfigure}[t]{0.45\textwidth}
			\includegraphics[width=\textwidth]{./imagens/dsc/DTAB_45p}
			\caption{45\% de ureia}
			\label{fig:DSC_DTAB_UR45}
		\end{subfigure} \qquad %
		\begin{subfigure}[t]{0.45\textwidth}
			\includegraphics[width=\textwidth]{./imagens/dsc/DTAB_40p}
			\caption{40\% de ureia}
			\label{fig:DSC_DTAB_UR40}
		\end{subfigure}
		\caption{Termogramas de soluções de DTAB 100, 200 e 300 \mM{}, em 45\% (\ref{fig:DSC_DTAB_UR45}) e 40\% (\ref{fig:DSC_DTAB_UR40}) de ureia}
		\label{fig:DSC_DTAB_UR_40-45}
	\end{figure}
	
%		
%		% todo: verificar se a notação (C|T|D) é boa
%		\begin{figure}[H]
%			\centering
%			\includegraphics[width=0.60\textwidth]{./imagens/dsc/Surf_100mm_45p}
%			\caption{Termogramas de soluções de (C|T|D)TAB 100 \mM{}, em 45\% de ureia}
%			\label{fig:DSC_Surf_100mm_45p}
%		\end{figure}
%	
%		\begin{figure}[H]
%			\centering
%			\includegraphics[width=0.60\textwidth]{./imagens/dsc/Surf_200mm_45p}
%			\caption{Termogramas de soluções de (C|T|D)TAB 200 \mM{}, em 45\% de ureia}
%			\label{fig:DSC_Surf_200mm_45p}
%		\end{figure}
%	
%		\begin{figure}[H]
%			\centering
%			\includegraphics[width=0.60\textwidth]{./imagens/dsc/Surf_300mm_45p}
%			\caption{Termogramas de soluções de (C|T|D)TAB 300 \mM{}, em 45\% de ureia}
%			\label{fig:DSC_Surf_300mm_45p}
%		\end{figure}

	\begin{figure}[H]
		\centering
		\includegraphics[width=0.60\textwidth]{./imagens/dsc/NaSal60}
		\caption{Termogramas de soluções de NaSal 60\mM{} e CTAB 100 \mM{}, em 35\%, 40\% e 45\% (m/m) de ureia}
		\label{fig:DSC_NaSal60}
	\end{figure}

	\begin{figure}[H]
		\centering
		\includegraphics[width=0.60\textwidth]{./imagens/dsc/NaSal100}
		\caption{Termogramas de soluções de NaSal 100\mM{} e CTAB 100 \mM{}, em 35\%, 40\% e 45\% (m/m) de ureia}
		\label{fig:DSC_NaSal100}
	\end{figure}

	\begin{figure}[H]
		\centering
		\includegraphics[width=0.60\textwidth]{./imagens/dsc/NaSal250}
		\caption{Termogramas de soluções de NaSal 250\mM{} e CTAB 100 \mM{}, em 35\%, 40\% e 45\% (m/m) de ureia}
		\label{fig:DSC_NaSal250}
	\end{figure}
%	
%		\begin{figure}[H]
%			\centering
%			\includegraphics[width=0.60\textwidth]{./imagens/dsc/NaSal35}
%			\caption{Termogramas de soluções de NaSal 60, 100 e 250\mM{} e CTAB 100 \mM{}, em 35\% (m/m) de ureia}
%			\label{fig:DSC_NaSal_Ur35}
%		\end{figure}
%	
%		\begin{figure}[H]
%			\centering
%			\includegraphics[width=0.60\textwidth]{./imagens/dsc/NaSal40}
%			\caption{Termogramas de soluções de NaSal 60, 100 e 250\mM{} e CTAB 100 \mM{}, em 40\% (m/m) de ureia}
%			\label{fig:DSC_NaSal_Ur40}
%		\end{figure}
%
%		\begin{figure}[H]
%			\centering
%			\includegraphics[width=0.60\textwidth]{./imagens/dsc/NaSal45}
%			\caption{Termogramas de soluções de NaSal 60, 100 e 250\mM{} e CTAB 100 \mM{}, em 45\% (m/m) de ureia}
%			\label{fig:DSC_NaSal_Ur45}
%		\end{figure}

		
	% (Figs. \ref{fig:DSC_CTAB_UR38-45}, \ref{fig:DSC_CTAB_UR40-45}, \ref{fig:DSC_TTAB_UR_40-45}, \ref{fig:DSC_DTAB_UR_40-45}, \ref{fig:DSC_NaSal60}, \ref{fig:DSC_NaSal100}, \ref{fig:DSC_NaSal250})
	
	Qualitativamente, pode-se observar que com o aumento de concentração de ureia, a temperatura de transição $T$ também aumenta. Além disso, com o aumento da concentração de surfactante, a área de transição $A$ aumenta e fica mais larga. Provavelmente, então, a concentração de ureia se relaciona com a estabilidade dos agregados, mas a concentração de surfactante, na faixa estudada, afeta somente a quantidade de agregados formados. 
	
	A temperatura de transição durante o aquecimento, $T_{aq}$, é sempre maior que a temperatura de transição no resfriamento, $T_{res}$. A largura a meia altura, $L$, segue a mesma tendência. Isso pode estar relacionado com a cinética de para desfazer e refazer os agregados. 
	
	Observa-se também que com o aumento na concentração de ureia, $L$ é pouco afetada, porém $T$ aumenta gradativamente, já a área $A$ varia dependendo da natureza do surfactante, com CTAB e TTAB sendo próximos entre si, e diferentes de DTAB.
	
	A adição de NaSal causa uma diminuição na temperatura de transição, especialmente visível em 250\mM{} de NaSal. Em vários casos não foi possível medir valores para as áreas de transição de resfriamento.
	
	As tabelas \ref{tab:DSC_temp_areas} e \ref{tab:DSC_temp_areas_NaSal} mostram as ição no aquecimento e resfriamento, obtidas pelo valor do máximo do pico, as áreas de transição e as larguras a meia altura dos picos de todos os experimentos realizados.
	
	Foram criados alguns gráficos para facilitar a visualização da dependência das propriedades dos termogramas em função da composição das soluções. A figura \ref{fig:DSC_propriedades_surf_40_45} relaciona as propriedades dos termogramas em função da concentração de surfactante, a figura \ref{fig:DSC_propriedades_Ur_38_45} relaciona as propriedades com a concentração de ureia e a figura \ref{fig:DSC_propriedades_NaSal} mostra o efeito de salicilato nas propriedades dos termogramas.

    \begin{table}[h]
        \IBGEtab%
        {\caption%
        	[Temperaturas, áreas e larguras a meia altura de CTAB, TTAB e DTAB em três concentrações de ureia.]%
        	{Temperaturas de transição ($T$/°C), áreas de transição por mol de surfactante ($A$/$J.mol^{-1}$) e largura a meia altura dos picos de transição ($L$/°C) dos ciclos de aquecimento (\emph{aq}) e resfriamento (\emph{res}) para três surfactantes (Surf.), CTAB, TTAB, DTAB, cujas concentrações estão em \mM, em várias concentrações \% (m/m) de ureia}
        \label{tab:DSC_temp_areas}}%
        {\begin{tabular}{ccccccccc}
            \toprule
			\% Ur. & Surf. & $C_{surf}$ & $T_{aq}$ & $T_{res}$ & $A_{aq}$ & $A_{res}$ & $L_{aq}$ & $L_{res}$ \\
			\midrule
			38     & CTAB  & 100        & 30,5     & 19,3      & 2,67     & 2,68      & 5,6      & 	3,0      \\
			39     & CTAB  & 100        & 32,5     & 25,2      & 2,28     & 2,30      & 6,4      & 	2,4      \\
			40     & CTAB  & 100        & 33,3     & 25,1      & 4,73     & 4,41      & 5,6      & 	3,0      \\
			41     & CTAB  & 100        & 34,1     & 27,2      & 1,74     & 1,77      & 5,3      & 	1,6      \\
			42     & CTAB  & 100        & 37,5     & 29,6      & 2,71     & 2,75      & 6,0      & 	1,9      \\
			43     & CTAB  & 100        & 38,7     & 31,1      & 2,10     & 2,10      & 5,4      & 	1,8      \\
			44     & CTAB  & 100        & 40,2     & 34,6      & 1,86     & 1,85      & 4,9      & 	1,6      \\
			45     & CTAB  & 100        & 41,0     & 34,6      & 2,04     & 2,02      & 4,6      & 	1,5      \\
			\midrule
			40     & CTAB  & 200        & 31,6     & 23,6      & 4,75     & 4,81      & 6,4      & 	2,4      \\
			45     & CTAB  & 200        & 41,6     & 34,9      & 1,72     & 1,92      & 10,3     & 3,5       \\
			40     & CTAB  & 300        & 31,7     & 24,8      & 4,84     & 4,79      & 5,3      & 	1,6      \\
			45     & CTAB  & 300        & 41,4     & 32,8      & 2,40     & 2,44      & 15,4     & 5,8       \\
			% 35   & CTAB  & 400        & 21,9     & 21,9      & 2,93     & 1,52      & 24,0     & 12,3      \\
			% 40   & CTAB  & 400        & 36,4     & 26,5      & 2,99     & 1,78      & 23,3     & 7,4       \\
			% 45   & CTAB  & 400        & 43,4     & 33,6      & 3,79     & 3,74      & 21,7     & 7,4       \\
			\midrule
			40     & DTAB  & 100        & 26,7     & 22,5      & 1,77     & 1,71      & 6,0      & 1,9       \\
			45     & DTAB  & 100        & 33,0     & 30,1      & 1,39     & 1,48      & 2,5      & 1,1       \\
			40     & DTAB  & 200        & 27,4     & 22,8      & 2,34     & 2,27      & 5,4      & 1,8       \\
			45     & DTAB  & 200        & 33,4     & 30,0      & 1,83     & 1,88      & 6,3      & 2,8       \\
			40     & DTAB  & 300        & 26,2     & 21,5      & 1,87     & 1,93      & 4,9      & 1,6       \\
			45     & DTAB  & 300        & 33,5     & 29,8      & 2,04     & 2,13      & 10,6     & 3,7       \\
			\midrule
			40     & TTAB  & 100        & 31,9     & 26,4      & 4,94     & 5,04      & 4,6      & 1,5       \\
			45     & TTAB  & 100        & 37,8     & 34,7      & 2,25     & 2,29      & 4,7      & 1,6       \\
			40     & TTAB  & 200        & 31,9     & 25,3      & 3,94     & 4,00      & 10,3     & 3,5       \\
			45     & TTAB  & 200        & 37,4     & 33,8      & 1,85     & 1,90      & 8,8      & 2,6       \\
			40     & TTAB  & 300        & 29,8     & 26,5      & 4,81     & 4,61      & 15,4     & 5,8       \\
			45     & TTAB  & 300        & 37,8     & 33,2      & 2,03     & 2,06      & 14,0     & 4,1       \\
            \bottomrule
            \end{tabular}}%
        {}
    \end{table}
    
    \begin{table}[H]
      \IBGEtab%
      {\caption{Temperaturas de transição ($T$/°C), áreas de transição por mol de surfactante ($A$/$J.mol^{-1}$) e largura a meia altura dos picos de transição ($L$/°C) dos ciclos de aquecimento (\emph{aq}) e resfriamento (\emph{res}) para CTAB com NaSal, cujas concentrações estão em \mM, em três concentrações \% (m/m) de ureia}
      \label{tab:DSC_temp_areas_NaSal}}%
        {\begin{tabular}{cccccccccc}
            \toprule
   			\% Ur. & Surf. & $C_{surf}$ & $C_{NaSal}$ & $T_{aq}$ & $T_{res}$ & $A_{aq}$ & $A_{res}$ & $L_{aq}$ & $L_{res}$\\
   			\midrule
   			35     & CTAB  & 100       & 60          & 22,0     & 14,6      & 5,07     & 3,48      & -        & - 		\\
   			40     & CTAB  & 100       & 60          & 27,6     & 22,0      & 6,67     & 6,18      & -        & - 		\\
   			45     & CTAB  & 100       & 60          & 42,0     & 23,5      & 5,66     & 4,06      & 15,0     & 3,0	\\
   			35     & CTAB  & 100       & 100         & 20,7     & -         & 5,00     & -         & 10,9     & - 		\\
   			40     & CTAB  & 100       & 100         & 27,1     & 22,9      & 7,03     & 5,82      & 15,3     & 	9,1	\\
   			45     & CTAB  & 100       & 100         & 29,9     & 26,0      & 5,51     & 4,87      & 11,0     & 	8,5	\\
   			35     & CTAB  & 100       & 250         & 22,8     & 12,7      & 6,05     & 2,98      & 8,5      & 	8,7	\\
   			40     & CTAB  & 100       & 250         & 23,3     & 11,8      & 4,55     & 2,60      & 7,1      & 	7,5	\\
   			45     & CTAB  & 100       & 250         & 29,1     & 27,0      & 9,99     & 5,93      & 12,3     & 	6,9	\\
   			\bottomrule
        \end{tabular}}%
            {}
        \end{table}

		\begin{figure}[H]
	 	\centering
	 	\begin{subfigure}[t]{0.45\textwidth}
	 		\includegraphics[width=\textwidth]{./imagens/dsc/L_40p_1_300mM_aq_res}
	 		\caption{$L$ a 40\% de ureia}
	 		\label{fig:DSC_L_40pUr}
	 	\end{subfigure} \qquad %
	 	\begin{subfigure}[t]{0.45\textwidth}
	 		\includegraphics[width=\textwidth]{./imagens/dsc/L_45p_1_300mM_aq_res}
	 		\caption{$L$ a 45\% de Ureia}
	 		\label{fig:DSC_L_45pUr}
	 	\end{subfigure}
 	
	 	\begin{subfigure}[t]{0.45\textwidth}
	 		\includegraphics[width=\textwidth]{./imagens/dsc/A_40p_1_300_aq_res}
	 		\caption{$A$ a 40\% de ureia}
	 		\label{fig:DSC_A_40pUr}
	 	\end{subfigure} \qquad %
	 	\begin{subfigure}[t]{0.45\textwidth}
	 		\includegraphics[width=\textwidth]{./imagens/dsc/A_45p_1_300_aq_res}
	 		\caption{$A$ a 45\% de Ureia}
	 		\label{fig:DSC_A_45pUr}
	 	\end{subfigure}
 	
	 	\begin{subfigure}[t]{0.45\textwidth}
	 		\includegraphics[width=\textwidth]{./imagens/dsc/T_40p_1_300_aq_res}
	 		\caption{$T$ a 40\% de ureia}
	 		\label{fig:DSC_T_40pUr}
	 	\end{subfigure} \qquad %
	 	\begin{subfigure}[t]{0.45\textwidth}
	 		\includegraphics[width=\textwidth]{./imagens/dsc/T_45p_1_300_aq_res}
	 		\caption{$T$ a 45\% de Ureia}
	 		\label{fig:DSC_T_45pUr}
	 	\end{subfigure}
 	
	 	\caption{Variação da largura a meia altura $L$ (\ref{fig:DSC_L_40pUr}, \ref{fig:DSC_L_45pUr}), área do pico $A$ (\ref{fig:DSC_A_40pUr}, \ref{fig:DSC_A_45pUr}), temperatura de transição $T$ (\ref{fig:DSC_T_40pUr}, \ref{fig:DSC_T_45pUr}) para os ciclos de aquecimento ($aq$) e resfriamento ($res$) de amostras de CTAB, DTAB e TTAB a 35\% e 40\% de ureia, de 100 a 300 \mM{} de surfactante}
	 	\label{fig:DSC_propriedades_surf_40_45}
	 \end{figure}
	
	\begin{figure}[H]
		\centering
		\begin{subfigure}[t]{0.60\textwidth}
			\includegraphics[width=\textwidth]{./imagens/dsc/L_100mM_aq_res}
			\caption{$L$ a 38\%---45\% de ureia}
			\label{fig:DSC_L_38_45pUr}
		\end{subfigure}
		
		\begin{subfigure}[t]{0.60\textwidth}
			\includegraphics[width=\textwidth]{./imagens/dsc/A_100mM_aq_res}
			\caption{$A$ a 38\%---45\% de ureia}
			\label{fig:DSC_A_38_45pUr}
		\end{subfigure} 
		
		\begin{subfigure}[t]{0.60\textwidth}
			\includegraphics[width=\textwidth]{./imagens/dsc/T_100mM_aq_res}
			\caption{$T$ a 38\%---45\% de ureia}
			\label{fig:DSC_T_38_45pUr}
		\end{subfigure}
		
		\caption{Variação da largura a meia altura $L$ (\ref{fig:DSC_L_38_45pUr}), área do pico $A$ (\ref{fig:DSC_A_38_45pUr}), temperatura de transição $T$ (\ref{fig:DSC_T_38_45pUr}) para os ciclos de aquecimento $aq$ e resfriamento $res$ de amostras de CTAB, DTAB e TTAB de 38\% a 45\% de ureia e 100 \mM{} de surfactante.}
		\label{fig:DSC_propriedades_Ur_38_45}
	\end{figure}
		
		\begin{figure}[H]
		\centering
		\begin{subfigure}[t]{0.45\textwidth}
			\includegraphics[width=\textwidth]{./imagens/dsc/A_NaSal_c_UR_aq_res}
			\caption{$A$ a 35\% de ureia}
			\label{fig:DSC_A_NaSal}
		\end{subfigure} \qquad %
		\begin{subfigure}[t]{0.45\textwidth}
			\includegraphics[width=\textwidth]{./imagens/dsc/T_NaSal_c_UR_aq_res}
			\caption{$T$ a 40\% de Ureia}
			\label{fig:DSC_T_NaSal}
		\end{subfigure}
		
		\caption{Variações da área do pico $A$ (\ref{fig:DSC_A_NaSal}), temperatura de transição $T$ (\ref{fig:DSC_T_NaSal}) para os ciclos de aquecimento ($aq$) e resfriamento ($res$) de amostras de CTAB e NaSal a 35\% a 40\% de ureia, de 60 a 250 \mM{} de NaSal}
		\label{fig:DSC_propriedades_NaSal}
	\end{figure}
		
\section{SAXS}

	Foram realizadas análises de SAXS para determinar a estrutura da fase branca. Foi avaliado o papel do comprimento da cadeia e da concentração de surfactante (Figs. \ref{fig:SAXS_ctabconc}, \ref{fig:SAXS_ttabconc}, \ref{fig:SAXS_dtabconc}).
	
	\begin{figure}[H]
		\centering
		\includegraphics[width=0.7\textwidth]{imagens/saxs/CTAB_conc}
		\caption{Curvas de SAXS de CTAB 100, 200 e 300 \mM{} e 40\% (m/m) de ureia. Os parâmetros de rede foram calculados com o primeiro pico.}
		\label{fig:SAXS_ctabconc}
	\end{figure}
	
	\begin{figure}[H]
		\centering
		\includegraphics[width=0.7\textwidth]{imagens/saxs/TTAB_conc}
		\caption{Curvas de SAXS de TTAB 100, 200 e 300 \mM{} e 40\% (m/m) de ureia. Os parâmetros de rede foram calculados com o primeiro pico.}
		\label{fig:SAXS_ttabconc}
	\end{figure}
	
	\begin{figure}[H]
		\centering
		\includegraphics[width=0.7\textwidth]{imagens/saxs/DTAB_conc}
		\caption{Curvas de SAXS de DTAB 100, 200 e 300 \mM{} e 40\% (m/m) de ureia. Os parâmetros de rede foram calculados com o primeiro pico.}
		\label{fig:SAXS_dtabconc}
	\end{figure}
	
		Para analisar a fase presente, é necessário indexar os picos. Isso é, determinar os valores de \q{} de seus máximos e observar a relação entre eles. Dependendo da relação, é possível afirmar que há um tipo de mesofase. Por exemplo, lamelas possuem espaçamento $1:2:3$, já fases hexagonais possuem espaçamento $1:\sqrt{3}:2:\sqrt{7}$. Em todas as curvas mostradas aqui, o espaçamento segue a regra $1:2:3$, ou seja, são observadas lamelas.
	
	Para se determinar o parâmetro de rede de lamelas, que é a distância de repetição, utiliza-se a equação \ref{eqn:SAXS_param_rede} utilizando o valor do pico de menor \q{} para o cálculo.
	
	% todo: colocar uma figura aqui mostrando o que é a distância de repetição
	% todo: colocar uma foto das lamelas
	
	\begin{equation}
	d = \frac{2\pi}{q}
	\label{eqn:SAXS_param_rede}
	\end{equation}
	
	As distâncias de repetição estão ilustradas junto com os valores dos picos nas figuras. É possível observar que as distâncias se tornam menores a medida que a concentração de surfactante aumenta, possivelmente porque mais surfactante resulta também em menos solvente. Além disso, é possível que a carga das micelas se torne menor devido à presença de mais contraíons em solução. Observa-se também que os picos se tornam melhor definidos e aparentemente menos largos com mais surfactante. É interessante notar que DTAB possui picos muito bem definidos em 300 \mM, e que sua distância de repetição é muito menor que os outros casos. A espessura dos picos está inversamente relacionada com o tamanho dos ``cristalitos'', de acordo com a equação de Scherrer. Logo, o aumento de concentração de surfactante também aumenta 

	% todo: encontrar a equação de Scherrer	
	% todo: pesquisar um termo melhor que cristalito.
	
	Também foi analisado o efeito da concentração de ureia em amostras de 300 \mM{} de CTAB (Fig. \ref{fig:SAXS_ctab300ur37-45}).
	
	\begin{figure}[H]
		\centering
		\includegraphics[width=0.5\textwidth]{imagens/saxs/CTAB300Ur37-45}
		\caption{Curvas de SAXS de CTAB 300 \mM{} em concentrações crescentes de ureia, com os parâmetros de rede calculados com o primeiro pico.}
		\label{fig:SAXS_ctab300ur37-45}
	\end{figure}
	
	Interessantemente, a adição de ureia aparenta diminuir o tamanho dos cristalitos, tanto que o trio de picos na faixa de \q{} acima de 1 nm\textsuperscript{-1} se junta, formando uma banda só em concentrações de ureia acima de 39\%. Observamos também que a distância de repetição é um pouco afetada pela adição de ureia, aumentando de 14,5 para 18,3 nm.
	
	Além disso, foram analisadas amostras a 50°C para obter informações sobre a fase transparente (Fig. \ref{fig:SAXS_surf50c}).
	
	\begin{figure}[H]
		\centering
		\includegraphics[width=0.7\textwidth]{imagens/saxs/Surf_50C}
		\caption{Curvas de SAXS de CTAB 100 e 300 \mM{} e TTAB 300 \mM{} com 45\% de ureia, a 50°C. As distâncias de correlação no pico foram calculadas.}
		\label{fig:SAXS_surf50c}
	\end{figure}

	Observa-se que os padrões de espalhamento se alteraram drasticamente, e que o padrão para CTAB e TTAB são bastante diferentes. Não foi possível medir DTAB pois havia muito pouco contraste.
	% todo: verificar porque não foi possível medir o DTAB de verdade.
	
	Essas soluções são pouco viscosas, transparentes e isotrópicas. Essas características sugerem que há alguma estrutura como micelas esféricas, micelas cilíndricas curtas ou vesículas nessas soluções. Seria interessante comparar os perfis de espalhamento da estruturas similares, e a maneira mais fácil de realizar isso é medir soluções com a mesma composição de surfactante, mas sem ureia. A Fig. \ref{fig:SAXS_micelas_esfericas} mostra os padrões de espalhamento de micelas em água para os três surfactantes, a 100, 200 e 300 \mM.
	
	\begin{figure}[H]
		\centering
		\includegraphics[width=0.7\textwidth]{imagens/saxs/micelas_esfericas}
		\caption{Curvas de SAXS de CTAB, TTAB e DTAB 100, 200 e 300 \mM{} em água, mostrando o perfil característico de micelas globulares carregadas.}
		\label{fig:SAXS_micelas_esfericas}
	\end{figure}  % todo: extrair os ``tamanhos'' dessas micelas, possivelmente por ajuste com o SUPERSAXS, e colocar em uma tabela aqui.
	
	% ref: Lutz-Bueno2017
	Observa-se o perfil característico de micelas carregadas, com um pico do correlação $q_{corr}$ em valores de \q{} menores e outro pico estrutural (\q{} maior). Quanto mais carregadas forem as micelas, maior será o pico de correlação. A Fig. \ref{fig:SAXS_ctab_q_corrs} mostra os picos de correlação. 
	
	\begin{figure}[H]
		\centering
		\includegraphics[width=0.7\textwidth]{imagens/saxs/CTAB_q_corrs}
		\caption{Curvas de SAXS de CTAB 100, 200 e 300 \mM{} em água, evidenciando os picos de correlação.}
		\label{fig:SAXS_ctab_q_corrs}
	\end{figure}

	Utilizando a equação \ref{eqn:SAXS_param_rede}, é possível encontrar a distância intermicelar média. A tabela X mostra as distâncias intermicelares calculadas por esse método.
	
	\begin{table}
		\IBGEtab%
		{\caption{Distâncias intermicelares $d_{im}$ de soluções de micelas esféricas de (C|T|D)TAB 100, 200, 300 \mM{} calculadas a partir de SAXS}
		\label{tab:SAXS_dim}}%
		{\begin{tabular}{p{3cm} p{3cm} p{0.1cm}}
			\toprule
			\centering Solução  & \centering $d_{im}$/nm & \\ 
			\midrule
			\centering CTAB 100  & \centering  12,5    & \\
			\centering CTAB 200  & \centering  10,4    & \\
			\centering CTAB 300  & \centering  9,57    & \\ 
			\midrule
			\centering TTAB 100  & \centering 9,71    & \\
			\centering TTAB 200  & \centering 8,26    & \\
			\centering TTAB 300  & \centering 7,55    & \\ 
			\midrule
			\centering DTAB 100  & \centering 9,44    & \\
			\centering DTAB 200  & \centering 7,14    & \\
			\centering DTAB 300  & \centering 6,31    & \\ 
			\bottomrule
		\end{tabular}}%
		{}
	\end{table} % todo: pensar se é possível estruturar essa tabela com Surf nas linhas e conc nas colunas, ou o contrário.
	% por algum motivo esta coluna extra na tabela é necessária.
	
	Observamos então que a distância intermicellar diminui com o aumento da concentração de surfactante, possivelmente porque a presença de uma maior quantidade de contraíons diminui a carga superficial micelar, logo a repulsão intermicelar diminui.
	
	% todo: verificar se essa explicação é a mesma do artigo da Viviane.
	
	%%%%%%%%%%
	Na seção X, mostra-se que há estruturas com um tamanho de aproximadamente 100-300 nm e outras com tamanhos de 3-5 nm. Devido às soluções serem transparentes, isotrópicas e pouco viscosas, e possuírem estruturas nessa faixa de tamanho, chegamos à conclusão que ocorre uma transição lamela-micela esférica. Logo, é interessante comparar o perfil de espalhamento das micelas em água, da figura \ref{fig:SAXS_micelas_esfericas}, com o perfil de micelas+lamelas da figura \ref{fig:SAXS_surf50c}. Observamos que as curvas de micelas globulares possuem um perfil característico de micelas carregadas, com um pico referente à repulsão eletrostática e outro referente ao tamanho das micelas.
	%%%%%%%%%%
	Caso a ureia removesse íons da superfície das micelas observaríamos um aumento da repulsão eletrostática entre as mesmas, que possivelmente poderia esconder o perfil do tamanho das micelas. A figura \ref{fig:SAXS_com_sem_ureia} mostra uma comparação de CTAB e TTAB com e sem ureia.
	
	\begin{figure}[H]
		\begin{subfigure}[t]{0.45\textwidth}
			\centering
			\includegraphics[width=\textwidth]{./imagens/saxs/micelas_CTAB_agua_ureia}
			\caption{CTAB}
			\label{fig:SAXS_CTAB_com_sem_ureia}
		\end{subfigure} \qquad %
		\begin{subfigure}[t]{0.45\textwidth}
			\centering
			\includegraphics[width=\textwidth]{./imagens/saxs/micelas_TTAB_agua_ureia}
			\caption{TTAB}
			\label{fig:SAXS_TTAB_com_sem_ureia}
		\end{subfigure}
		\caption{Curvas de SAXS de CTAB (\ref{fig:SAXS_CTAB_com_sem_ureia}) e TTAB (\ref{fig:SAXS_TTAB_com_sem_ureia}) em água a 25°C e em ureia 45\% a 50°C}
		\label{fig:SAXS_com_sem_ureia}
	\end{figure}

	% todo: colocar uma boa discussão sobre a diferença nos perfis de (C|T|D)TAB com o aumento da concentração de surfactante. 
	% todo: checar o artigo que estuda micelas esféricas carregadas e sem carga. Lutz-Bueno2017.
	% todo: colocar o label da seção e comparar com os resultados dos tamanhos de DLS observados.

	Os padrões de espalhamento para os surfactantes sólidos estão na Fig. \ref{fig:SAXS_surf_sólido}.

	\begin{figure}[H]
		\centering
		\includegraphics[width=0.7\textwidth]{imagens/saxs/surfactante_solido}
		\caption{Curvas de SAXS de CTAB, TTAB e DTAB sólidos, a 25°C}
		\label{fig:SAXS_surf_sólido}
	\end{figure}  % todo: extrair os ``tamanhos'' dessas micelas, possivelmente por ajuste com o SUPERSAXS, e colocar em uma tabela aqui.
	
\section{DLS}
\section{Reologia do sólido}
\section{Entalpia de interação de ureia com surfactante}
