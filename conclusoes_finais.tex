\chapter{Conclusões finais}

	Os resultados reológicos mostraram que foi necessário considerar parâmetros além do índice de refração \(n\), inicialmente proposto, para explicar os fenômenos observados. A constante de Hamaker \(A\) possui, além de \(n\), uma contribuição da constante dielétrica \(\varepsilon\), parâmetro que também está relacionado com a capacidade de solvatação de cargas. Esse parâmetro foi necessário para entender como a ureia afeta os resultados reológicos e calorimétricos, pois esse é o único aditivo que aumenta esse parâmetro, o que pode ter levado à dessorção de íons salicilato e contraíons da superfície micelar.
	
	Apesar disso, somente esses dois parâmetros não foram suficientes para explicar as variações observadas. Glicerina 60\% e sacarose 50\% possuem praticamente os mesmos valores de ambos, mas perfis de viscosidade e de calorimetria totalmente diferentes. Foi proposto um terceiro, o parâmetro de Gordon \(G\), relacionado com a estruturação do solvente. Esse parâmetro, apesar de não oferecer a melhor descrição da energia coesiva, já foi utilizado na literatura para explicar fenômenos em outros sistemas de autoassociação. Foi visto que havia uma boa correlação entre o efeito dos aditivos estudados e seus respectivos parâmetros de Gordon. Por exemplo, a sacarose não afetou o valor de \(G\), e tanto o perfil reológico quanto calorimétrico não foram afetados. Já o \BD{} possui a maior queda de \(G\), e os efeitos mais fortes na viscosidade, apesar de que a calorimetria foi pouco afetada.
	
	Essa divergência entre o comportamento reológico e calorimétrico sugere que é necessário considerar também fatores adicionais, relacionados à estrutura dos aditivos, como sua capacidade de interagir com a superfície ou o interior da micela. Esses efeitos de superfície dificultaram a explicação completa dos perfis de calorimetria de formação de micelas gigantes, principalmente as variações de \DHwlm. Porém, os três parâmetros empregados foram suficientes para explicar a maior parte das variações em \cmc{} e \DHmic. Essas correlações mostram também que dois fenômenos díspares, como a reologia de micelas gigantes no regime semidiluído, e a formação de micelas em concentrações muito diluídas estão relacionados, e são afetados pelas características do solvente de maneiras similares, mas não idênticas.
	
	Logo, a hipótese inicial está incompleta, e é necessário considerar alterações não só na constante de Hamaker, como o efeito da estruturação do solvente e a interação dos aditivos com a superfície micelar, duas características relacionadas com a estrutura de cada aditivo. Essa abordagem diferenciada e resultado inédito garantiram a publicação desse estudo numa revista de qualidade da área.
	
	Em etapas futuras, esse estudo poderá ser expandido utilizando-se mais aditivos. Por exemplo, poderia ser proposta uma série de álcoois, similares aos estudados, com domínios polares de tamanhos diferentes. Como todos são líquidos, a energia coesiva poderia ser estimada pela entalpia de vaporização, e seu efeito comparado com o parâmetro de Gordon. Poderiam ser estudados aditivos que aumentam \(\varepsilon\), como aminas secundárias, e comparar seu efeito com a ureia. Como sistema de comparação, poderiam ser utilizados surfactantes que formam lamelas, pois essas estruturas são afetadas mais fortemente por variações na constante de Hamaker do que micelas gigantes, assim separando os efeitos desse fenômeno dos efeitos do solvente. Além disso, poderiam ser propostos parâmetros numéricos adicionais, por exemplo, relacionados à interação dos aditivos com a superfície micelar. Estudos de simulação poderiam fornecer detalhes adicionais sobre como a estrutura do solvente afeta as micelas gigantes.
	
	Notou-se também que a ureia levou à formação de estruturas lamelares em altas concentrações. As técnicas utilizadas, apesar de não apresentar definitivamente como é a estrutura dessas lamelas e qual é o papel da ureia, possibilitam a expansão em outras áreas. Por exemplo, poderia ser utilizada Ressonância Magnética Nuclear (RMN) para estudar a localização da ureia nos agregados, e Microscopia de Transmissão Eletrônica em temperaturas criogênicas (Cryo-TEM) para determinar os tamanhos dos domínios formados.
	
	Seguindo o tema de correlacionar propriedades diferentes, foi possível mostrar que a calorimetria, viscosidade e intensidade de fluorescência de salicilato possuem correlações entre si, então as alterações em uma das propriedades levam à mudanças nas outras propriedades também. Com isso em mente, esperava-se que o decaimento na intensidade de fluorescência estaria relacionado ao aumento da viscosidade, logo, ao crescimento das micelas. Porém, o decaimento presente nesse experimento não foi comprovadamente ligado ao crescimento micelar, podendo ter origem na fonte de luz utilizada. Esses resultados abrem caminho para estudos futuros, onde poderão ser utilizadas fontes de luz mais específicas, menos intensas, para medir o espalhamento, e estudos com uma montagem experimental mais sofisticada e precisa.
	
	Por último, através de SAXS resolvido no tempo observou-se que a cinética de crescimento das micelas ocorria na faixa de tempo entre 35 e 65 ms. O maior desafio desse experimento foi justamente a escolha de um sistema com bom contraste. Com isso em mente, seria possível continuar essa área de pesquisa utilizando uma combinação de surfactante e cossoluto fluorescente com um contraste muito melhor. O cossoluto poderia ser escolhido com base em estudos na literatura onde se observa as características em comum dos aditivos que formam micelas gigantes. Dessa maneira, seria possível aliar os estudos de ITC, de fluorescência e de SAXS resolvido no tempo para determinar a cinética de crescimento de micelas.
	
	