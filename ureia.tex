\chapter{Efeito da ureia}
	\section{Motivação}
	A ureia demonstrou um comportamento que divergiu bastante dos outros aditivos. Por esse motivo, ela será estudada um pouco mais profundamente. Porém, a ação do salicilato de sódio não receberá muito enfoque, para simplificar o sistema. Portanto, foi estudado principalmente o efeito da ureia em soluções de CTAB, TTAB e DTAB, com concentrações diferentes de ureia e de surfactante.
	
	Ocorre a formação de um precipitado esbranquiçado em soluções de surfactante em concentrações maiores que 35\% de ureia. Isso ocorre a temperatura ambiente. Quando a solução é aquecida acima de cerca de 35ºC, ela se torna transparente. Esse comportamento foi estudado, variando-se o surfactante, sua concentração, e a concentração de ureia. Desses sistemas, foram estudadas as características térmicas, a estrutura da mesofase, e a reologia da fase esbranquiçada.
	
	\section{Calorimetria diferencial de varredura (DSC)}
		% todo: colocar o número da figura
		Foram preparadas soluções de ureia, em várias concentrações, com três surfactantes (CTAB, TTAB e DTAB), em três concentrações. Os termogramas resultantes foram organizados em figuras de modo a facilitar comparações. A tabela \ref{tab:refs_DSC} lista as comparações realizadas e em quais figuras estão.
		% todo: trocar por ibgetable

		\begin{longtable}[h]{l c c}
			\caption{Comparações realizadas e suas respectivas figuras} \\
			\label{tab:refs_DSC} \\
			\toprule
			\centering
			Conc. Surfactante \mM     & \% Ureia		& Figura 			\\
			\midrule
%            \endfirsthead
			\endhead
			CTAB 100	  & 38--45			& \ref{fig:DSC_CTAB_UR38-45}	\\*
			CTAB 100, 200, 300	& 45, 40	& \ref{fig:DSC_CTAB_UR40-45}	\\*
			TTAB 100, 200, 300	& 45, 40	& \ref{fig:DSC_TTAB_UR_40-45}	\\*
			DTAB 100, 200, 300	& 45, 40	& \ref{fig:DSC_DTAB_UR_40-45}	\\*
			CTAB, TTAB, DTAB 100	& 45	& \ref{fig:DSC_Surf_100mm_45p}	\\*
			CTAB, TTAB, DTAB 200	& 45	& \ref{fig:DSC_Surf_200mm_45p}	\\*
			CTAB, TTAB, DTAB 300	& 45	& \ref{fig:DSC_Surf_300mm_45p}	\\*
			\midrule
			CTAB 100 NaSal 60	& 35, 40, 45	& \ref{fig:DSC_NaSal60}		\\*
			CTAB 100 NaSal 100	& 35, 40, 45	& \ref{fig:DSC_NaSal100}	\\*
			CTAB 100 NaSal 250	& 35, 40, 45	& \ref{fig:DSC_NaSal250}	\\*
			CTAB 100 NaSal 60, 100, 250 & 35 	& \ref{fig:DSC_NaSal_Ur35}  \\*
			CTAB 100 NaSal 60, 100, 250 & 40	& \ref{fig:DSC_NaSal_Ur40}  \\*
			CTAB 100 NaSal 60, 100, 250	& 45	& \ref{fig:DSC_NaSal_Ur45}  \\*
			\bottomrule

		\end{longtable}
		
		\begin{figure}[h]
			\centering
			\includegraphics[width=0.75\textwidth]{./imagens/dsc/CTAB_porc_ur}
			\caption{Termogramas de soluções de CTAB 100 \mM{} em concentrações crescentes de ureia, de 38\% m/m a 45\% m/m}
			\label{fig:DSC_CTAB_UR38-45}
		\end{figure}
		
		\begin{figure}[h]
			\begin{subfigure}[h]{0.45\textwidth}
				\centering
				\includegraphics[width=\textwidth]{./imagens/dsc/CTAB_45p}
				\caption{45\% de ureia}
				\label{fig:DSC_CTAB_UR45}
			\end{subfigure} \qquad %
			\begin{subfigure}[h]{0.45\textwidth}
				\centering
				\includegraphics[width=\textwidth]{./imagens/dsc/CTAB_40p}
				\caption{40\% de ureia}
				\label{fig:DSC_CTAB_UR40}
			\end{subfigure}
			\caption{Termogramas de CTAB 100, 200 e 300 \mM{} em soluções em 45\% (\ref{fig:DSC_CTAB_UR45}) e 40\% (\ref{fig:DSC_CTAB_UR40})}
			\label{fig:DSC_CTAB_UR40-45}
		\end{figure}
	
		% todo: colocar o título da figura do 40% do lado direito
		% todo: verificar se o tamanho da fonte está correto.
		\begin{figure}[h]
			\centering
			\begin{subfigure}[t]{0.45\textwidth}
				\includegraphics[width=\textwidth]{./imagens/dsc/TTAB_45p}
				\caption{45\% de ureia}
				\label{fig:DSC_TTAB_UR45}
			\end{subfigure} \qquad %
			\begin{subfigure}[t]{0.45\textwidth}
				\includegraphics[width=\textwidth]{./imagens/dsc/TTAB_40p}
				\caption{40\% de ureia}
				\label{fig:DSC_TTAB_UR40}
			\end{subfigure}
			\caption{Termogramas de soluções de TTAB 100, 200 e 300 \mM{}, em 45\% (\ref{fig:DSC_TTAB_UR45}) e 40\% (\ref{fig:DSC_TTAB_UR40}) de ureia}
			\label{fig:DSC_TTAB_UR_40-45}
		\end{figure}
		
		\begin{figure}[h]
			\centering
			\begin{subfigure}[t]{0.45\textwidth}
				\includegraphics[width=\textwidth]{./imagens/dsc/DTAB_45p}
				\caption{45\% de ureia}
				\label{fig:DSC_DTAB_UR45}
			\end{subfigure} \qquad %
			\begin{subfigure}[t]{0.45\textwidth}
				\includegraphics[width=\textwidth]{./imagens/dsc/DTAB_40p}
				\caption{40\% de ureia}
				\label{fig:DSC_DTAB_UR40}
			\end{subfigure}
			\caption{Termogramas de soluções de DTAB 100, 200 e 300 \mM{}, em 45\% (\ref{fig:DSC_DTAB_UR45}) e 40\% (\ref{fig:DSC_DTAB_UR40}) de ureia}
			\label{fig:DSC_DTAB_UR_40-45}
		\end{figure}
		
		
		% todo: verificar se a notação (C|T|D) é boa
		\begin{figure}[h]
			\centering
			\includegraphics[width=0.45\textwidth]{./imagens/dsc/Surf_100mm_45p}
			\caption{Termogramas de soluções de (C|T|D)TAB 100 \mM{}, em 45\% de ureia}
			\label{fig:DSC_Surf_100mm_45p}
		\end{figure}
	
		\begin{figure}[h]
			\centering
			\includegraphics[width=0.45\textwidth]{./imagens/dsc/Surf_200mm_45p}
			\caption{Termogramas de soluções de (C|T|D)TAB 200 \mM{}, em 45\% de ureia}
			\label{fig:DSC_Surf_200mm_45p}
		\end{figure}
	
		\begin{figure}[h]
			\centering
			\includegraphics[width=0.45\textwidth]{./imagens/dsc/Surf_300mm_45p}
			\caption{Termogramas de soluções de (C|T|D)TAB 300 \mM{}, em 45\% de ureia}
			\label{fig:DSC_Surf_300mm_45p}
		\end{figure}
	
		\begin{figure}[h]
			\centering
			\includegraphics[width=0.45\textwidth]{./imagens/dsc/NaSal60}
			\caption{Termogramas de soluções de NaSal 60\mM{} e CTAB 100 \mM{}, em 35\%, 40\% e 45\% (m/m) de ureia}
			\label{fig:DSC_NaSal60}
		\end{figure}
	
		\begin{figure}[h]
			\centering
			\includegraphics[width=0.45\textwidth]{./imagens/dsc/NaSal100}
			\caption{Termogramas de soluções de NaSal 100\mM{} e CTAB 100 \mM{}, em 35\%, 40\% e 45\% (m/m) de ureia}
			\label{fig:DSC_NaSal100}
		\end{figure}

		\begin{figure}[h]
			\centering
			\includegraphics[width=0.45\textwidth]{./imagens/dsc/NaSal250}
			\caption{Termogramas de soluções de NaSal 250\mM{} e CTAB 100 \mM{}, em 35\%, 40\% e 45\% (m/m) de ureia}
			\label{fig:DSC_NaSal250}
		\end{figure}
	
		\begin{figure}[h]
			\centering
			\includegraphics[width=0.45\textwidth]{./imagens/dsc/NaSal35}
			\caption{Termogramas de soluções de NaSal 60, 100 e 250\mM{} e CTAB 100 \mM{}, em 35\% (m/m) de ureia}
			\label{fig:DSC_NaSal_Ur35}
		\end{figure}
	
		\begin{figure}[h]
			\centering
			\includegraphics[width=0.45\textwidth]{./imagens/dsc/NaSal40}
			\caption{Termogramas de soluções de NaSal 60, 100 e 250\mM{} e CTAB 100 \mM{}, em 40\% (m/m) de ureia}
			\label{fig:DSC_NaSal_Ur40}
		\end{figure}

		\begin{figure}[h]
			\centering
			\includegraphics[width=0.45\textwidth]{./imagens/dsc/NaSal45}
			\caption{Termogramas de soluções de NaSal 60, 100 e 250\mM{} e CTAB 100 \mM{}, em 45\% (m/m) de ureia}
			\label{fig:DSC_NaSal_Ur45}
		\end{figure}

		% todo: colocar aqui uma tabela com as áreas de transição, dependendo da concentração de surfactante utilizado (mol), e uma estimativa da largura a meia altura de cada transição.
		% todo: verificar se há bons métodos para determinar essa linha base. Senão, utilizar um valor medido na mão mesmo.
		
		Qualitativamente, podemos observar que com o aumento de concentração de ureia, a temperatura de transição também aumenta (Figs. \ref{fig:DSC_CTAB_UR38-45}, \ref{fig:DSC_CTAB_UR40-45}, \ref{fig:DSC_TTAB_UR_40-45}, \ref{fig:DSC_DTAB_UR_40-45}, \ref{fig:DSC_NaSal60}, \ref{fig:DSC_NaSal100}, \ref{fig:DSC_NaSal250}). Além disso, vemos que CTAB e TTAB possuem temperaturas semelhantes de transição, maiores que DTAB (Figs. \ref{fig:DSC_Surf_100mm_45p}, \ref{fig:DSC_Surf_200mm_45p}, \ref{fig:DSC_Surf_300mm_45p}). As áreas de transição, e a largura dos picos são, também, proporcionais à concentração de surfactante utilizado. A adição de NaSal causa uma diminuição na temperatura de transição, especialmente visível em 250\mM{} de NaSal.
		% todo: trocar por IBGEtable
		\begin{longtable}[h]{cccccccccc}
			\caption{Temperaturas de transição ($T$/°C), áreas de transição por mol de surfactante ($A$/$J.mol^{-1}$) e largura a meia altura dos picos de transição ($L$/°C) dos ciclos de aquecimento (\emph{aq}) e resfriamento (\emph{res}) para três surfactantes (Surf.), CTAB, TTAB, DTAB, cujas concentrações estão em \mM, em várias concentrações \% (m/m) de ureia}\\
            \label{tab:DSC_temp_areas}\\
			\toprule
			\centering
			\% Ur. & Surf. & $C_{surf}$ & $C_{NaSal}$ &
			$T_{aq}$ & $T_{res}$ & $A_{aq}$ & $A_{res}$ & $L_{aq}$ & $L_{res}$\\
			\midrule
			\endhead
			38 & CTAB & 100 & 0 & 30.5 & 19.3 & 2.67 & 2.68 & 5.6 &	3.0\\*
			39 & CTAB & 100 & 0 & 32.5 & 25.2 & 2.28 & 2.30 & 6.4 &	2.4\\*
			40 & CTAB & 100 & 0 & 33.3 & 25.1 & 4.73 & 4.41 & 5.6 &	3.0\\*
			41 & CTAB & 100 & 0 & 34.1 & 27.2 & 1.74 & 1.77 & 5.3 &	1.6\\*
			42 & CTAB & 100 & 0 & 37.5 & 29.6 & 2.71 & 2.75 & 6.0 &	1.9\\*
			43 & CTAB & 100 & 0 & 38.7 & 31.1 & 2.10 & 2.10 & 5.4 &	1.8\\
			44 & CTAB & 100 & 0 & 40.2 & 34.6 & 1.86 & 1.85 & 4.9 &	1.6\\
			45 & CTAB & 100 & 0 & 41.0 & 34.6 & 2.04 & 2.02 & 4.6 &	1.5\\
			40 & CTAB & 200 & 0 & 31.6 & 23.6 & 4.75 & 4.81 & 6.4 &	2.4\\
			45 & CTAB & 200 & 0 & 41.6 & 34.9 & 1.72 & 1.92 & 10.3 & 3.5\\
			40 & CTAB & 300 & 0 & 31.7 & 24.8 & 4.84 & 4.79 & 5.3 &	1.6\\
			45 & CTAB & 300 & 0 & 41.4 & 32.8 & 2.40 & 2.44 & 15.4 & 5.8\\
			35 & CTAB & 400 & 0 & 21.9 & 21.9 & 2.93 & 1.52 & 24.0 & 12.3\\
			40 & CTAB & 400 & 0 & 36.4 & 26.5 & 2.99 & 1.78 & 23.3 & 7.4\\
			45 & CTAB & 400 & 0 & 43.4 & 33.6 & 3.79 & 3.74 & 21.7 & 7.4\\
			\midrule
			40 & DTAB & 100 & 0 & 26.7 & 22.5 & 1.77 & 1.71 & 6.0 & 1.9\\
			45 & DTAB & 100 & 0 & 33.0 & 30.1 & 1.39 & 1.48 & 2.5 & 1.1\\
			40 & DTAB & 200 & 0 & 27.4 & 22.8 & 2.34 & 2.27 & 5.4 &	1.8\\
			45 & DTAB & 200 & 0 & 33.4 & 30.0 & 1.83 & 1.88 & 6.3 &	2.8\\
			40 & DTAB & 300 & 0 & 26.2 & 21.5 & 1.87 & 1.93 & 4.9 &	1.6\\
			45 & DTAB & 300 & 0 & 33.5 & 29.8 & 2.04 & 2.13 & 10.6 & 3.7\\
			\midrule
			40 & TTAB & 100 & 0 & 31.9 & 26.4 & 4.94 & 5.04 & 4.6 &	1.5\\
			45 & TTAB & 100 & 0 & 37.8 & 34.7 & 2.25 & 2.29 & 4.7 &	1.6\\
			40 & TTAB & 200 & 0 & 31.9 & 25.3 & 3.94 & 4.00 & 10.3 & 3.5\\
			45 & TTAB & 200 & 0 & 37.4 & 33.8 & 1.85 & 1.90 & 8.8 &	2.6\\
			40 & TTAB & 300 & 0 & 29.8 & 26.5 & 4.81 & 4.61 & 15.4 & 5.8\\
			45 & TTAB & 300 & 0 & 37.8 & 33.2 & 2.03 & 2.06 & 14.0 & 4.1\\
			\midrule
			35 & CTAB & 100 & 60 & 22.0 & 14.6 & 5.07 & 3.48 & - & -\\
			40 & CTAB & 100 & 60 & 27.6 & 22.0 & 6.67 & 6.18 & - & -\\
			45 & CTAB & 100 & 60 & 42.0 & 23.5 & 5.66 & 4.06 & 15.0 & 3.0\\
			35 & CTAB & 100 & 100 & 20.7 & - & 5.00 & - & 10.9 & -\\
			40 & CTAB & 100 & 100 & 27.1 & 22.9 & 7.03 & 5.82 & 15.3 &	9.1\\
			45 & CTAB & 100 & 100 & 29.9 & 26.0 & 5.51 & 4.87 & 11.0 &	8.5\\*
			35 & CTAB & 100 & 250 & 22.8 & 12.7 & 6.05 & 2.98 & 8.5 &	8.7\\*
			40 & CTAB & 100 & 250 & 23.3 & 11.8 & 4.55 & 2.60 & 7.1 &	7.5\\*
			45 & CTAB & 100 & 250 & 29.1 & 27.0 & 9.99 & 5.93 & 12.3 &	6.9\\*
			\bottomrule

		\end{longtable}
		
		
		
	\section{SAXS}
	\section{DLS}
	\section{Reologia do sólido}
	\section{Entalpia de interação de ureia com surfactante}