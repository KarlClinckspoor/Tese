% todo: pensar em colocar uma parte falando sobre o método dos mínimos quadrados
\part{Teoria}
	\chapter{Reologia}
		\section{Fundamentos}
			\subsection{Fluídos Newtonianos}
			\subsection{Sólidos Hookeanos}
			\subsection{Fluidos viscoelásticos}
		\section{Reologia oscilatória}
			\subsection{Aquisição de dados}
			\subsection{Modelo de Maxwell}
			\subsection{Modelos mais complexos}
		\section{Curvas de Fluxo}
			\subsection{Modelos de curvas de fluxo}
			% todo: colocar a seção dos apêndices onde eu descrevo as curvas aqui, e depois só referenciar o apêndice para a implementação técnica dos modelos
	\chapter{Calorimetria de titulação isotérmica}
		\section{Fundamentos}
			\subsection{Aquisição de dados}
		\section{Calorimetria de micelas esféricas}
		\section{Calorimetria de micelas gigantes}
		\section{Termodinâmica de micelização}
	\chapter{SAXS}
		\section{Fundamentos}
		\section{Modelagem}
			\subsection{Esferas}
			\subsection{Micelas esféricas}
			\subsection{Micelas gigantes}
			\subsection{Visualização dos parâmetros}
			\subsection{Indexação de picos}
	\chapter{Fluorescência}
		\section{Fundamentos}
			\subsection{Diagramas}
			\subsection{Rendimento quântico}
			\subsubsection{Lei de X (não importa onde incide para fluorescência)}
	\chapter{Análise Multivariada}
		\section{Técnicas de classificação}
			\subsection{Normalização dos dados}
			\subsection{PCA}
			\subsection{HCA}
		\section{Técnicas de regressão}
			\subsection{Regressão Multivariada}
			\subsection{PCR}
			\subsection{PLS}