\chapter{Efeito da ureia}
	\section{Motivação}
	A ureia demonstrou um comportamento que divergiu bastante dos outros aditivos. Por esse motivo, ela será estudada um pouco mais profundamente. Porém, a ação do salicilato de sódio não receberá muito enfoque, para simplificar o sistema. Portanto, foi estudado principalmente o efeito da ureia em soluções de CTAB, TTAB e DTAB, com concentrações diferentes de ureia e de surfactante.
	
	Ocorre a formação de um precipitado esbranquiçado em soluções de surfactante em concentrações maiores que 35\% de ureia. Isso ocorre a temperatura ambiente. Quando a solução é aquecida acima de cerca de 35ºC, a solução se torna transparente. Esse comportamento foi estudado, variando-se o surfactante, sua concentração, e a concentração de ureia. Desses sistemas, foram estudadas as características térmicas, a estrutura da mesofase, e a reologia da fase esbranquiçada formada.
	
	\section{Calorimetria diferencial de varredura (DSC)}
		% todo: colocar o número da figura
		Foram preparadas soluções de concentrações crescentes de surfactante e ureia. A Figura X mostra como os perfis de calorimetria são afetadas pelas variações do comprimento da cadeia do surfactante, em três concentrações de surfactante, em 40\% de ureia. A Figura Y mostra as mesmas variações de composição, mas com 45\% de ureia.
		
		% Figura X
		
		% Figura Y
		
		A figura Z mostra o efeito da concentração de ureia em soluções de CTAB 100 \mM{}, a figura Z+1 de CTAB 300 \mM{} e a figura Z+2 de CTAB 400 \mM.
		
		As figuras W, W+1 e W+2 mostram os efeitos de salicilato de sódio, em três concentrações de NaSal, em concentrações de 35\%, 40\% e 45\% (m/m) de ureia, respectivamente.
		
		%%%%%%%%% MG %%%%%%%%%%%%%%%%%%%%%
		% CTAB 100 NaSal 60: 35, 40, 45%
		% CTAB 100 NaSal 100: 35, 40, 45%
		% CTAB 100 NaSal 260: 35, 40, 45%
		%%%%%%%%%%%%%%%%%%%%%%%%%%%%%%%%%%
		
		% Efeito do tamanho da cadeia do surfactante	
		% CTAB, TTAB, DTAB; 100, 200, 300 a 45% ureia
		% CTAB, TTAB, DTAB; 100, 200, 300 a 40% ureia
		
		% Efeito da concentração de ureia
		% CTAB100 de 38% a 45% de ureia
		% CTAB300 de 38% a 45% de ureia (mandar fazer)
		% CTAB400 de 35%, 40%, 45%
		
	\section{SAXS}
	\section{DLS}
	\section{Reologia do sólido}
	\section{Entalpia de interação de ureia com surfactante}