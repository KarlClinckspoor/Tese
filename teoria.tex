% todo: pensar em colocar uma parte falando sobre o método dos mínimos quadrados
\part{Teoria}
	\chapter{Reologia}
		\section{Fundamentos}
		
		% todo: Pensar se eu devo colocar que vem do grego rheos, corrente/rio/fluxo e logia, estudo/ciência
		% todo: pensar se eu devo falar que foi cunhado pelo Prof. Bingham em 1929
		A reologia é a área ciência que estuda o fluxo e a deformação da matéria. Para causar um fluxo ou deformação, é necessário que uma força externa seja aplicada ao corpo. Reagindo à essa força, o material se comporta de tal maneira que algumas de suas características estruturais podem ser inferidas. Isso é feito intuitivamente por qualquer pessoa, por exemplo, ao apertar uma fruta para determinar sua firmeza e assim, se ela está apropriada para consumo ou não, para desagrado do vendedor.  % todo: pensar se eu deixo essa piadinha aqui ou não
		
		No campo coloidal, a reologia é utilizada para estudar como os corpos coloidais estão dispostas no meio, suas interações entre si e com o meio, e como fluem mediante a força externa. Por exemplo, soluções de micelas gigantes são altamente viscosas, pois as cadeias das micelas se entrelaçam e ramificam, então existe um mecanismo para oferecer resistência à força aplicada. Já soluções de micelas esféricas possuem baixa viscosidade, pois o tamanho das micelas é pequeno, e não existe esse mecanismo. A resistência se deve principalmente ao solvente, nesse caso.
		
		A viscosidade pode ser definida, de maneira pouco rigorosa, como a resistência ao fluxo de um material. Então água possui baixa viscosidade, já mel, uma solução concentrada de açúcar, possui alta viscosidade. Porém, quando tenta-se aplicar essa definição para outros tipos de materiais, aparecem problemas. Manteiga mantém seu formato, ao contrário de mel, que sempre flui, mas é muito mais fácil passar manteiga num pedaço de pão do que mel. Qual seria mais viscoso? % todo: mencionar amido de milho funciona?
		
		% todo: checar o tipo de emulsão da manteiga
		O comportamento diferente da manteiga é causado por sua microestrutura coloidal. Manteiga é uma emulsão água em óleo, ou seja, há gotículas de água estabilizadas pelas proteínas do leite dispersas e compactadas num meio contínuo de óleo. As forças atrativas entre as gotículas, e a concentração relativa pequena de óleo, faz com que as gotículas consigam se estruturar no meio, impedindo o fluxo sob uma força externa bastante fraca como a gravidade. Porém, as forças entre as gotículas podem ser rompidas quando uma força externa um pouco mais forte é aplicada. Após o rompimento das interações intergotículas, o fluxo se torna fácil. Se a força for removida, as interações são reformadas e o material volta a assumir sua consistência característica.
		
		Portanto, a viscosidade da manteiga dependeu da força que estava sendo aplicada. Em forças baixas, o material aparentava ser altamente viscoso, já em forças maiores, o material aparentava ser pouco viscoso. Não existe um valor único de viscosidade que pode ser atribuído à manteira, da mesma maneira que é feito com água. Somente é possível estabelecer viscosidades aparentes dependentes da força aplicada.
		
		Do lado oposto à viscosidade, no campo reológico, existe a elasticidade. Materiais elásticos, ao invés de fluir, se deformam reversivelmente. A estrutura interna desses materiais permite que energia seja armazenada em torções e distensões. Quando a força externa é removida, a energia é liberada e o material volta ao seu formato inicial. Caso a energia aplicada supere as forças que estruturam o material, por exemplo, ligações covalentes ou interações intermoleculares, o material acaba fluindo ou quebrando. Exemplos clássicos de materiais elásticos são: molas, borrachas, rochas, madeira. A constante de elasticidade é a grandeza análoga à viscosidade, para materiais elásticos. Essa constante está relacionada à força necessária para causar uma deformação no material.

		Retomando o exemplo anterior, em baixos valores de força, a manteiga conseguia se manter estruturada. Nessa região de forças bastante pequenas, a manteiga consegue armazenar energia em sua estrutura interna. De acordo com as definições apresentadas, a manteiga pode se comportar tanto como um material elástico quanto um material viscoso. Esse tipo de material recebe o nome de material viscoelástico.
		
		\subsection{Número de Deborah}
		
		É possível relacionar materiais viscosos, elásticos e viscoelásticos através do número de Deborah (Eq. \ref{eqn:Deborah}), que relaciona o tempo de relaxação do material (\(\tau_{\text{rel}}\)) com o tempo de observação (\(t\)).
		
		\begin{equation}
			D_e = \dfrac{\tau_{\text{rel}}}{t}
			\label{eqn:Deborah}
		\end{equation}
		
		O tempo de relaxação é, como o nome diz, o tempo que um material leva para dissipar uma tensão aplicada em si. Quando a força é aplicada por um tempo maior que o tempo de relaxação do material, a tensão é dissipada e o material flui. De maneira oposta, caso a tensão seja aplicada por um tempo bastante curto, comparativamente, não há tempo para o material perder a energia, e mantém seu formato.
		
		% todo: achar uma ref pra isso
		Por exemplo, um arco de madeira utilizado para caça ou para guerra devia ser armazenado sem sua corda. Caso fosse, a força do arco cairia com o tempo e a eficácia do arco diminuía. Isso ocorre porque as fibras da madeira que compõe o arco, se tensionadas por muito tempo, conseguem deslizar umas pelas outras para dissipar a tensão. Logo, o arco, notavelmente um material elástico no uso cotidiano, se comporta como um fluído se tensionado por muito tempo.
		
		Água, no outro extremo, é um material que se comporta quase sempre como um líquido. Porém, a água age como um sólido quando um objeto impacta sobre ela, a uma velocidade suficientemente grande. Isso ocorre porque as moléculas de água não tem tempo suficiente para se deslocar, e a água é um fluído incompressível, resultando numa resistência forte ao fluxo.
		
		Soluções de micelas gigantes combinam ambos os comportamentos elástico e viscoso. Isso é observado claramente no efeito de recuo (\emph{recoil}), quando uma solução de micelas gigantes é agitada circularmente. Inicialmente, observa-se a solução seguindo o sentido do fluxo, possivelmente através das bolhas formadas pela agitação. Quando a agitação é cessada, o fluxo continua por um tempo, devido à sua inércia, depois para e começa a fluir no sentido contrário (recuo). As cadeias de micelas gigantes conseguem interagir entre si e se entrelaçar de tal modo que um pouco da energia da agitação é armazenada. Quando a agitação cessa, essa energia consegue ser liberada, o que ocorre no sentido oposto à agitação, então o recuo é observado. Depois do recuo, a solução para de se movimentar pois toda a energia foi dissipada. A escala de tempo para a perda de energia e para o \emph{recoil} são aproximadamente iguais, logo o número de Deborah desse fluído é próximo de 1.
		
		A correlação entre o número de Deborah e o comportamento do material pode ser resumido pela Eq. \ref{eqn:Deborah_casos}
		
		\begin{equation}
			D_e
			\begin{cases}
				\gg 1     & \textrm{Elástico}      \\
				\approx 1 & \textrm{Viscoelástico} \\
				\ll 1     & \textrm{Viscoso}
			\end{cases}
			\label{eqn:Deborah_casos}
		\end{equation}
		
		O arco em uso possui \(D_e\gg1\), pois seu tempo de relaxação é bastante longo comparado com a ação de atirar uma flecha, mas \(D_e\ll1\) quando armazenado, o que pode durar meses. Dessa maneira, o arco e os outros materiais citados se comportam como materiais tanto elásticos quanto viscosos, dependendo do tempo de análise \(t\). O nome desse número vem de uma passagem bíblica, \emph{Os montes deslizaram diante do senhor} (Juízes 5:5). Mesmo as montanhas, que tradicionalmente não se movem, acabam deslizando mediante tempos de observação infinitos.
		
			\subsection{Fluídos Newtonianos}
			
			% Começar a estabelecer o formalismo. Colocar termos para tensão, cisalhamento, taxa de cisalhamento.
					
			Os fluídos Newtonianos obedecem a lei de Newton da viscosidade
			\begin{equation}
				\tau = \eta\dot{\gamma}
				\label{eqn:Newton}
			\end{equation}
	
			% Colocar um exemplo de uma curva de fluxo para fluído Newtoniano
				
			\subsection{Sólidos Hookeanos}
			
			% Colocar a equação de Hooke, mostrar o que é G
			
			% Colocar um exemplo de curva de fluxo Hookeana, e as regiões onde não há mais o comportamento. Fazer um paralelo, ou deixar implícito, com o exemplo da parte de fundamentos, sobre sobrepor a energia interna.
			
			\subsection{Fluidos viscoelásticos}
			
			% Micelas gigantes são fluidos viscoelásticos por causa da estrutura
			% Mostrar o formato de curvas de fluxo de micelas gigantes, comparando com as hookeanas e newtonianas
			% Mostrar de onde é obtido o valor de viscosidade no platô
			% Mostrar os modelos que podem ser usados para fluidos viscoelásticos.
			% Fazer referência ao apêndice, onde tem alguns exemplos de curvas.
			% Mencionar como que os dados são adquiridos nas curvas de fluxo
			
		\section{Reologia oscilatória}
			\subsection{Aquisição de dados}
			\subsection{Modelo de Maxwell}
			\subsection{Modelos mais complexos}
			% todo: colocar a seção dos apêndices onde eu descrevo as curvas aqui, e depois só referenciar o apêndice para a implementação técnica dos modelos
	\chapter{Calorimetria de titulação isotérmica}
		\section{Fundamentos}
			\subsection{Aquisição de dados}
		\section{Calorimetria de micelas esféricas}
		\section{Calorimetria de micelas gigantes}
		\section{Termodinâmica de micelização}
	\chapter{SAXS}
		\section{Fundamentos}
		\section{Modelagem}
			\subsection{Esferas}
			\subsection{Micelas esféricas}
			\subsection{Micelas gigantes}
			\subsection{Visualização dos parâmetros}
			\subsection{Indexação de picos}
	\chapter{Fluorescência}
		\section{Fundamentos}
			\subsection{Diagramas}
			\subsection{Rendimento quântico}
			\subsubsection{Lei de X (não importa onde incide para fluorescência)}
	\chapter{Coloides}
	% todo: falar sobre a constante de Hamaker e outras coisas importantes para a discussão
		\section{Atração coloidal}
		% Falar sobre como as atrações de vdW em coloides são aditivas, e sobre o problema enfrentado.
		% Falar sobre os estudos do físico para tentar contornar esse problema.
		% Falar sobre DLVO
		\section{Constante de Hamaker}
		% Colocar a equação e falar como que se obtêm os valores
	\chapter{Micelas gigantes}
		\label{chap:micelas_gigantes}
		\section{Crescimento de micelas}
		% Falar sobre o diagrama de fases de micelas
		% Procurar um pouco, na tese da Danila e em teses anteriores do grupo, sobre o que falar a mais de micelas
		% todo: completar esta seção
		As micelas crescem.
		% Falar sobre os tempos de relaxação vistos por Hoffmann por birrefringência elétrica
		\section{Termodinâmica de micelas}
		% Falar sobre a energia das pontas. Pesquisar sobre o que falar.
		% Fazer uma breve introdução sobre ITC e o que se observa.
		% Falar sobre o ITC e como ele consegue enxergar os processos de micelização
		% todo: fazer esta parte
		As micelas tem pontas com energia superior.
		\section{Comportamento reológico}
		% Falar sobre como identificar visualmente micelas gigantes, como o recoil
		% Falar sobre as várias estruturas que elas possuem, e como a concentração de salicilato afeta as estruturas
		% Falar sobre as G', G'' e o modelo de Maxwell.
		% Falar sobre o tempo de relaxação micelar.
		% Falar sobre os sticky contacts que o Hoffmann tanto gosta.
		\section{Efeito do solvente}
		\label{sec:efeito_solvente}
		% Falar mais profundamente sobre os estudos do Hoffmann
		% Falar sobre a hidrofilicidade da superfície micelar, o t3 de relaxação e a relação com os perfis de viscosidade.
		
		
%	\chapter{Análise Multivariada}
%		\section{Técnicas de classificação}
%			\subsection{Normalização dos dados}
%			\subsection{PCA}
%			\subsection{HCA}
%		\section{Técnicas de regressão}
%			\subsection{Regressão Multivariada}
%			\subsection{PCR}
%			\subsection{PLS}