\documentclass[a4paper, 10pt]{book}
\usepackage[a4paper,margin=2.0cm]{geometry}

%%%% Fontes e língua %%%%
\usepackage{fontspec}%for XeLaTeX, selecting multiple fonts
\usepackage{polyglossia}%for XeLaTeX, enables multiple languages.
\setmainlanguage{brazil}
\PolyglossiaSetup{brazil}{indentfirst=true}
\setotherlanguages{english}

%\setmainfont[Ligatures=TeX]{Latin Modern Roman}
%\setsansfont[Ligatures=TeX]{Latin Modern Sans}
%\setmonofont{Latin Modern Mono}

\setmainfont[Ligatures=TeX]{Latin Modern Roman}
\setsansfont[Ligatures=TeX]{Latin Modern Roman}
\setmonofont{Latin Modern Mono} 


%\usepackage[brazilian]{babel}
%\usepackage[utf8]{inputenc}
%\usepackage[T1]{fontenc}

\usepackage{csquotes}
\usepackage{amsmath, amsfonts, amssymb}
\usepackage{xfrac}
%\mathchardef\period=\mathcode`.               % Muda . para , e usa como sep. decimal
%\DeclareMathSymbol{.}{\mathord}{letters}{"3B} % Muda . para , e usa como sep. decimal
%\renewcommand{\baselinestretch}{1.15}
\usepackage[compact]{titlesec}
\usepackage{setspace}
\setstretch{1.3}
\usepackage{multicol}
%%%%%%%%%%%%%%%%%

\usepackage{xcolor}
%\usepackage{enumerate}  % listas com numeração diferente
\usepackage{enumitem}

%%%% Gráficos, posicionamento de tabelas
\usepackage{graphicx}
\usepackage{float}
\floatstyle{plaintop} % tabelas com legenda na parte de cima
\restylefloat{table}  % tabelas com legenda na parte de cima

\usepackage{longtable}
\usepackage{booktabs}

\usepackage[labelsep=period]{caption} % figuras com subfiguras
\usepackage{subcaption}  % figuras com subfiguras
\graphicspath{{.}}

%%%% Headers e footers %%%%
\usepackage{fancyhdr}
\pagestyle{fancy}

%\let\Subsectionmark\subsectionmark
%\def\subsectionmark#1{\def\Subsectionname{#1}\Subsectionmark{#1}}

\fancyhead{}
\fancyfoot{}
%\fancyhead[L]{Karl Jan Clinckspoor}
%\fancyhead[L]{Exame de Qualificação}
%\fancyhead[R]{\nouppercase{\rightmark}}
%\fancyhead[R]{\thesubsection~\quad~\nouppercase{\Subsectionname}}

\fancyfoot[C]{\thepage}
%\lhead{\footnotesize Exame de qualificação}
%\rhead{\footnotesize \sectionmark}
%\cfoot{\thepage}
\renewcommand{\footrulewidth}{0pt}
\renewcommand{\headrulewidth}{0pt}
\setlength{\headheight}{15pt} 
%%%%%

\newcommand{\mM}{mmol.L\textsuperscript{-1}}
\newcommand{\Sal}{Sal\textsuperscript{--}}

%%%%

\usepackage{hyperref}
\usepackage{blindtext}


\begin{document}
	
	\tableofcontents
	
Elaboração geral das áreas e conteúdo da tese.
	
\part{Introdução}
	\chapter{Surfactantes}
		\section{Polaridade}
		\section{Parâmetro de empacotamento}
		\section{Mesofases}
		\section{Aditivos}
	\chapter{Micelas gigantes}
		\section{Crescimento de micelas}
		\section{Termodinâmica de micelas}
		\section{Modelos de comportamento reológico}
		\section{Perfis de viscosidade}
	\chapter{Inspirações para o projeto}
		\chapter{Estudos de Hoffmann sobre micelas e lamelas}
		\chapter{Estudos de Pedersen sobre cinética}
	\chapter{Objetivos}

\part{Teoria}
	\chapter{Reologia}
		\section{Fundamentos}
			\subsection{Fluídos Newtonianos}
			\subsection{Sólidos Hookeanos}
			\subsection{Fluidos viscoelásticos}
		\section{Reologia oscilatória}
			\subsection{Aquisição de dados}
			\subsection{Modelo de Maxwell}
			\subsection{Modelos mais complexos}
		\section{Curvas de Fluxo}
			\subsection{Modelos de curvas de fluxo}
	\chapter{Calorimetria de titulação isotérmica}
		\section{Fundamentos}
			\subsection{Aquisição de dados}
		\section{Calorimetria de micelas esféricas}
		\section{Calorimetria de micelas gigantes}
		\section{Termodinâmica de micelização}
	\chapter{SAXS}
		\section{Fundamentos}
		\section{Modelagem}
			\subsection{Esferas}
			\subsection{Micelas esféricas}
			\subsection{Micelas gigantes}
			\subsection{Visualização dos parâmetros}
			\subsection{Indexação de picos}
	\chapter{Fluorescência}
		\section{Fundamentos}
			\subsection{Diagramas}
			\subsection{Rendimento quântico}
			\subsubsection{Lei de X (não importa onde incide para fluorescência)}
	\chapter{Análise Multivariada}
		\section{Técnicas de classificação}
			\subsection{Normalização dos dados}
			\subsection{PCA}
			\subsection{HCA}
		\section{Técnicas de regressão}
			\subsection{Regressão Multivariada}
			\subsection{PCR}
			\subsection{PLS}

\part{Materiais e Métodos}
	\chapter{Reagentes}
	\chapter{Reologia}
		\section{Preparo das amostras}
		\section{Tratamento de dados de reologia oscilatória}
		\section{Tratamento de dados de curvas de fluxo}
	\chapter{Calorimetria de titulação isotérmica}
		\section{Preparo das amostras}
		\section{Tratamento de dados}
	\chapter{SAXS}
		\section{Aquisição de dados}
			\subsection{LNLS}
			\subsection{Grenoble}
			\subsection{Stopped-flow}
		\section{Tratamento de dados}
			\subsection{Subtração do ``branco''}
			\subsection{Média das curvas de cinética}
			\subsection{Ajuste das curvas pelo software superSAXS}
	\chapter{Fluorescência}
		\section{Aquisição de dados}
			\subsection{Determinação da absorção e emissão}
			\subsection{Fluorescência estática}
			\subsection{Fluorescência resolvida no tempo}
				\subsubsection{Programa LabView}
		\section{Tratamento de dados}
			\subsection{Filtro Savitzky-Golay}
	\chapter{Técnicas adicionais}
		\section{Calorimetria diferencial de varredura}
		\section{Espalhamento dinâmico de luz}
		\section{Tensiometria}
		
\part{Efeito dos aditivos hidrofílicos}
	\chapter{Resultados}
		\section{Efeitos dos aditivos na reologia}
			\subsection{Glicerina}
			\subsection{Sacarose}
			\subsection{DMSO}
			\subsection{1,3BD}
			\subsection{Ureia}
		\section{Efeito dos aditivos na calorimetria de micelas gigantes}
		\section{Efeito dos aditivos na calorimetria de micelização}
	\chapter{Parâmetros a ser estudados}
		\subsection{Índice de refração}
		\subsection{Constante dielétrica}
		\subsection{Parâmetro de Gordon}
		\subsection{Interação dos aditivos com a superfície micelar}
		\subsection{Decomposição em propriedades fundamentais}
	\chapter{Correlações entre os parâmetros e as propriedades}
		\subsection{Reologia}
		\subsection{Calorimetria}
	\chapter{Efeito da Ureia}
		\section{Calorimetria diferencial de varredura}
		\section{SAXS}
		\section{DLS}
		\section{Reologia do sólido}
		\section{Entalpia de interação de ureia com surfactante}
	
\part{Cinética de crescimento}
	\chapter{SAXS resolvido no tempo}
	\chapter{Fluorescência resolvida no tempo}
	
\part{Projetos menores}
	\chapter{Estudo sobre regiões Maxwellianas nos perfis de viscosidade}
	\chapter{Comparação de ITC de MG em dois sentidos opostos}
	
	
\part{Contribuições para outros projetos}
	\chapter{Muco}
		\section{Breve descrição do projeto}
		\section{Contribuição}
			\subsection{Determinação de uma metodologia}
			\subsection{Tratamento de dados}
		\section{Resultado da colaboração}
	\chapter{Previsão de temperaturas de fusão de triacilglicerídeos}
		\section{Breve descrição}
		\section{Contribuição}

\part{Anexos}
	\chapter{Instalação e uso de Python}
		\section{Curso de Python}
	\chapter{Descrição extensa do modelo de SAXS de micelas gigantes}
	\chapter{Descrição e uso do software de tratamento de curvas de fluxo}
	\chapter{Softwares miscelâneos para tratamento de dados}
	
\end{document}