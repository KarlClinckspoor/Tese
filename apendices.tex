% TODO: reorganizar este apêndice para fazer mais sentido

\begin{apendicesenv}
	\partapendices
	
	\chapter{Descrição matemática do modelo de micelas gigantes}
	
	\section{Introdução e motivação}
	% TODO: colocar a seção onde o modelo foi utilizado
%	
%	Esta seção mostrará as equações do modelo de micelas gigantes, utilizado neste trabalho. O modelo foi criado pelo Prof. Jan Skov Pedersen, em Fortran 77, e disponibilizado para o aluno para realizar os ajustes das curvas obtidas no ESRF. O aluno então transcreveu o código de Fortran para Python, uma linguagem mais clara, e criou um pequeno programa interativo que relaciona o modelo com seus parâmetros. O programa consegue também comparar uma curva teórica com dados experimentais, de modo a fornecer um bom chute inicial para o ajuste das curvas.
%	
%	% todo: colocar uma imagem do modelo em ação
%	
%	Uma das motivações para realizar essa tarefa foi a grande dificuldade de nosso grupo utilizar SAXS para estudar micelas. Não há especialistas na região que conhecem modelagem de micelas gigantes. Além disso, escrever um modelo computacional baseando-se somente as equações matemáticas é uma tarefa muito difícil. O modelo computacional, apesar de menos compacto em sua notação, é mais legível, o que também facilita o entendimento de novos alunos.
	
	\section{Resumo do modelo}
	
	O modelo descreve cadeias alongadas caroço-casca (\emph{core-shell}) de Kratky-Porod, considerando volume excluído, com interações intercadeias modeladas pelo modelo PRISM (\emph{Polymer Reference Interaction Site Model}). No total, a equação de intensidade de espalhamento $I$ em função do vetor de espalhamento \q ($I(q)$, Eq. \ref{eqn:saxs_MG_superficial}) possui 13 parâmetros, descritos na tabela \ref{tab_ap:simbolos}.
	
	\begin{equation}
	I = f(q, scale, d_{head}, r_{core}, \rho_{rel}, \sigma, back, L, k_L, \varepsilon, D_{CQ}, \nu_{RPA}, SC_{pow}, exp_{pow})
	\label{eqn:saxs_MG_superficial}
	\end{equation}
	% TODO: colocar refs às equações que definem alguns desses termos
	\begin{longtable}[c]{r p{12cm}}
		\toprule
		Símbolo 			& Descrição        						\\
		\midrule
		$I$					& Intensidade de RX espalhado			\\
		\q					& Vetor de espalhamento					\\
		\midrule
		$scale$				& Fator de escala						\\
		$d_{head}$			& Espessura do \emph{shell}				\\
		$r_{core}$			& Raio do \emph{core}					\\
		$\rho_{rel}$		& Diferença de densidade eletrônica entre \emph{core} e \emph{shell} \\
		$\sigma$			& Fator de \emph{smearing}, o quão definido é o limite entre regiões \\
		$back$				& Constante referente ao \emph{background} \\
		$L$					& Comprimento de contorno das cadeias 	\\
		$k_L$				& Comprimento de \emph{Kuhn} das cadeias, igual ao dobro do comprimento de correlação \\ % TODO: Achar o que é esse comprimento exatamente
		$\varepsilon$			& Excentricidade radial das micelas		\\
		$D_{CQ}$			& Distância de correlação das micelas 	\\
		$\nu_{RPA}$			& Fator de concentração					\\
		\midrule
		$SC_{pow}$			& Fator de escala (preexponencial) da exponencial em baixo q\\
		$exp_{pow}$			& Fator exponencial, relativo à inclinação na escala log\\
		\bottomrule
		\caption{Símbolos e parâmetros utilizados no modelo, e seus significados}
		\label{tab_ap:simbolos} 
	\end{longtable}
	
	A equação geral do modelo, e a descrição de seus fatores, estão descritos na Eq.\ref{eqn:saxs_MG_geral} e na Tab. \ref{tab_ap:fatores_geral}.
	
	\begin{equation}
	I = \frac{scale\left(F_{KPchain_{ExV}}F_{rod_{CS}}\right)}{1 + \nu_{RPA} F_{sphere}\left( D_{CQ}\right) F_{KPchain_{ExV}}} + back + scale_{pow}^{-exp_{pow}}
	\label{eqn:saxs_MG_geral}
	\end{equation}
	
	\begin{longtable}[c]{r p{12cm}}
		\toprule
		Termo 			& Descrição        						\\
		\midrule
		$F_{KPchain_{ExV}}$  & Fator forma de cadeias de Kratky-Porod com volume excluído \\
		$F_{rod_{CS}}$		 & Fator forma da seção transversão de um bastão	\\
		$F_{sphere}(D_{CQ})$ & Fator forma de uma esfera, cujo raio é a distância de correlação \\
		\bottomrule
		\caption{Parâmetros da equação \ref{eqn:saxs_MG_geral}}
		\label{tab_ap:fatores_geral} 
	\end{longtable}
	
	Já o modelo do PRISM é descrito pela Eq. \ref{eqn:saxs_PRISM}. Note a similaridade com a Eq \ref{eqn:saxs_MG_geral}.
	
	\begin{equation}
	I_{PRISM}= \frac{\varphi V_{mic}F_{wc}(q)F_{cs}(q)}{1 + \nu F_{rod}(qL_{c(q)})F_{wc}(q)}
	\label{eqn:saxs_PRISM}
	\end{equation}
	
	\begin{longtable}[c]{r p{12cm}}
		\toprule
		Termo 			& Descrição        						\\
		\midrule
		$\varphi$		& Fração volumétrica \\ % TODO: checar
		$V_{mic}$		& Volume da micela   \\
		$F_{wc}$		& Fator forma de uma \emph{wormlike chain} \\
		$F_{cs}$		& Fator forma de uma seção transversal de cilindro \\ % TODO: checar se é cilindro
		$F_{rod}$		& Fator forma de um bastão infinitamente longo \\
		$L_{c(q)}$		& $=6\xi$, comprimento característico \\
		$\xi$			& Comprimento de correlação da função $c(q) \approx F_{rod}$ \\
		\bottomrule
		\caption{Termos da equação \ref{eqn:saxs_PRISM}}
		\label{tab_ap:PRISM_geral} 
	\end{longtable}
	
	A partir disso, podemos começar a adentrar nos termos.
	
	\section{Descrição detalhada do modelo}
	
	O modelo será dividido em duas partes, uma referente à cadeia micelar, $F_{wc}$ e outra referente à seção transversal da cadeia, $F_{cs}$.
	
	\subsection{Fator forma das cadeias \emph{wormlike}, $F_{wc}$}
	
	\begin{equation}
	F_{wc} = \left[\left(1 - \chi\right)F_{chain_{ExV}} + \chi F_{rod}\right]\Gamma
	\label{eqn:Fwc}
	\end{equation}
	
	% TODO: colocar a equação do \Gamma
	A equação \ref{eqn:Fwc} pode ser simplificada dependendo da faixa de \q. A região de \q intermediária precisa ser descrita pelo termo $\chi$ (Eq. \ref{eqn:chi}) e corrigida por $\Gamma$. Esses parâmetros são obtidos por simulações de Monte Carlo.
	
	\begin{equation}
	F_{wc} \left\{
	\begin{matrix}
	q\ baixo: F_{wc} \approx F_{chain_{ExV}} \\
	q\ alto: F_{wc} \approx F_{rod} \\
	\end{matrix} \right.
	\end{equation}
	
	%	\begin{longtable}[c]{r p{12cm}}
	%		\toprule
	%		Termo 			& Descrição        						\\
	%		\midrule
	%		$\chi$			& Região de \emph{crossover} \\
	%		$\Gamma$		& Correção da região de crossover. \\
	%		\bottomrule
	%		\caption{Termos da equação \ref{eqn:Fwc}}
	%		\label{tab_ap:Fwc} 
	%	\end{longtable}
	
	\subsubsection{Fator de correção $\chi$}
	O termo $\chi$ é descrito pela equação \ref{eqn:chi}, que por sua vez é dependente da equação \ref{eqn:xi}.
	
	\begin{equation}
	\chi = \exp{\xi^{-5}}
	\label{eqn:chi}
	\end{equation}
	
	% todo: achar o significado do termo b
	\begin{equation}
	\xi = q k_L\left(\frac{\pi b}{1,103L}\right)^{3/2}\left(\frac{\left<R_g^2\right>}{k_L^2}\right)^{1,282}
	\label{eqn:xi}
	\end{equation}
	
	\noindent onde $\left<R_g^2\right>$ é a média do \emph{ensemble} do quadrado do raio de giro das cadeias, no modelo.
	
	\subsubsection{Fator forma de cadeias com volume excluído, $F_{chain_{ExV}}$}
	
	O termo $F_{chain_{ExV}}$ possui a seguinte forma (Eq. \ref{eqn:FchainExV})
	
	% TODO: verificar se o \nu aqui é o \nu_RPA
	\begin{multline}
	F_{chain_{ExV}} = w(qR_g)F_{Debye}(q,L,k_L) + \left[1 - w(q R_g)\right] \\ \left[C_1(q R_g)^{\frac{1}{\nu}} + C_2(q R_g)^{-\frac{2}{\nu}} + 
	C_3(q R_g)^{-\frac{3}{\nu}}\right]
	\label{eqn:FchainExV}
	\end{multline}
	
	% todo: incluir aqui uma tabela com os termos e as equações
	
	O termo $F_{Debye}$, por sua vez, é dado pela Eq. \ref{eqn:fdebye}.
	\begin{equation}
	F_{Debye} = 2 \left(\frac{e^{-u} + u - 1}{u^2}\right)
	\label{eqn:fdebye}
	\end{equation}
	
	\noindent onde $u = R_g^2q^2$. $R_g$ é a raiz quadrada do raio de giro médio ao quadrado, $R_g = \left<R_g^2\right>^{1/2}$, considerando o volume excluído. Por sua vez, esse valor é dado pela Eq. \ref{eqn:Rg2}
	
	% Achar o que significam os termos faltantes aqui.
	\begin{equation}
	\left<R_g^2\right> = \alpha \left(\frac{L}{k_L}\right)^2\left<R_g^2\right>_0
	\label{eqn:Rg2}
	\end{equation}
	
	O termo $w$ é uma equação empírica, da forma: (Eq \ref{eqn:w})
	
	\begin{equation}
	w(x) = \frac{\left[1 + \frac{\tanh(x-C_4)}{C_5}\right]}{2}
	\label{eqn:w}
	\end{equation}
	
	As constantes $C_1$, $C_2$, $C_3$, $C_4$ e $C_5$ foram obtidas a partir de um ajuste, e estão na tabela \ref{tab_ap:C1C5}.
	
	\begin{longtable}[c]{r l}
		\toprule
		Constante 			& Valor \\
		\midrule
		$C_1$			&  1,220	\\
		$C_2$			&  0,4288	\\
		$C_3$			&  --1,651	\\
		$C_4$			&  1,523	\\
		$C_5$			&  0,1477 	\\						
		\bottomrule
		\caption{Constantes}
		\label{tab_ap:C1C5} 
	\end{longtable}
	
	\subsubsection{Fator de correção $\Gamma$}
	
	O fator de correção $\Gamma$ (Eq. \ref{eqn:Gamma}) é dependente de dois conjuntos de constantes, $A$ (Eq. \ref{eqn:Ai}) e B (Eq. \ref{eqn:Bi}) determinadas empiricamente (Tab \ref{tab_ap:AiBi}).
	
	\begin{equation}
	\Gamma\left( q,L,k_{L} \right) = 1 + \left( 1 - \chi \right)\sum_{i = 2}^{5}{A_{i}\xi^{i}} + \chi\sum_{i = 0}^{2}{B_{i}\xi^{- i}}
	\label{eqn:Gamma}
	\end{equation}
	
	\begin{equation}
	A_{i} = \sum_{j = 0}^{2}{a_{1}\left( i,j \right)\left( \frac{L}{k_{L}} \right)^{- j}\exp\left( - \frac{10k_{L}}{L} \right)} + \sum_{j = 1}^{2}{a_{2}\left( i,j \right)\left( \frac{L}{k_{L}} \right)^{j}\exp\left( - \frac{2L}{k_{L}} \right)}
	\label{eqn:Ai}
	\end{equation}
	
	\begin{equation}
	B_{i} = \sum_{j = 0}^{2}{b_{1}\left( i,j \right)\left( \frac{L}{k_{L}} \right)^{- j}\ } + \sum_{j = 1}^{2}{b_{2}\left( i,j \right)\left( \frac{L}{k_{L}} \right)^{j}\exp\left( - \frac{2L}{k_{L}} \right)}
	\label{eqn:Bi}
	\end{equation}
	
	\begin{longtable}[c]{r l | r l | r l | r l}
		\toprule
		\endhead
		$a_1$(2,0) & --0.1222 & $a_2$(2,1) & 0.1212 & $b_1$(0,0) &
		--0.0699 & $b_2$(0,1) & --0.5171\\
		$a_1$(3,0) & 0.3051 & $a_2$(3,1) & --0.4169 & $b_1$(1,0) & --0.09
		& $b_2$(1,1) & --0.2028\\
		$a_1$(4,0) & --0.0711 & $a_2$(4,1) & 0.1988 & $b_1$(2,0) &
		0.2677 & $b_2$(2,1) & --0.3112\\
		$a_1$(5,0) & 0.0584 & $a_2$(5,1) & 0.3435 & $b_1$(0,1) & 0.1342
		& $b_2$(0,2) & 0.6950\\
		$a_1$(2,1) & 1.761 & $a_2$(2,2) & 0.0170 & $b_1$(1,1) & 0.0138 &
		$b_2$(1,2) & --0.3238\\
		$a_1$(3,1) & 2.252 & $a_2$(3,2) & --0.4731 & $b_1$(2,1) & 0.1898
		& $b_2$(2,2) & --0.5403\\
		$a_1$(4,1) & --1.291 & $a_2$(4,2) & 0.1869 & $b_1$(0,2) & --0.2020
		& &\\
		$a_1$(5,1) & 0.6994 & $a_2$(5,2) & 0.3350 & $b_1$(1,2) & --0.0114
		& &\\
		$a_1$(2,2) & --26.04 & & & $b_1$(2,2) & 0.0123 & &\\
		$a_1$(3,2) & 20.00 & & & & & &\\
		$a_1$(4,2) & 4.382 & & & & & &\\
		$a_1$(5,2) & 1.594 & & & & & &\\
		\bottomrule
		\caption{Constantes utilizadas para o cálculo de $\Gamma$}
		\label{tab_ap:AiBi}
	\end{longtable}
	
	\subsubsection{Fator forma de um cilindro $F_{rod}$}
	
	O fator forma de um cilindro segue a equação \ref{eqn:Frod}.
	
	\begin{equation}
	F_{rod}(q, L) = \frac{2Si(qL)}{qL} - \frac{4\sin^2\frac{qL}{2}}{q^2L^2}
	\label{eqn:Frod}
	\end{equation}
	
	\noindent onde $Si$ é a função-integral de seno (Eq. \ref{eqn:Si})
	
	\begin{equation}
	Si(x) = \int_0^x \frac{\sin t}{t}dt
	\label{eqn:Si}
	\end{equation}
	
	% TODO: verificar se F_CS é de fator da seção de um cilindro. CS é cross section mesmo.
	\subsection{Fator forma da seção transversal de um cilindro $F_{cs}$}
	
	O fator forma da seção transversal de um cilindro é descrito pela equação \ref{eqn:Fcs}. Seus parâmetros se encontram na tabela \ref{tab_ap:Fcs}
	
	\begin{equation}
	F_{\text{cs}} = \frac{2}{\pi}\int_{0}^{\frac{\pi}{2}}%
	%
	\left[ \left(\rho_{S} - \rho_{w} \right) \frac{2J_1 \left( qR_{s}\left( \varepsilon,\theta \right) \right)}{qR_{s}\left( \varepsilon,\theta \right)} % 
	%
	+  %
	%
	\frac{\pi\varepsilon R_c^2}{\pi\varepsilon R_s^2}\left( \rho_c - \rho_s \right)	%
	%
	\frac{2J_1\left( qR_{c}\left( \varepsilon,\theta \right) \right)}{qR_{c}\left( \varepsilon,\theta \right)}\  \right]^2 d\theta
	\label{eqn:Fcs}
	\end{equation}
	
	% Todo: padronizar essas equações
	% Todo: verificar se o \varepsilon é a excentricidade
	% Todo: verificar se Rc e Rs são funções de \varepsilon e \theta
	\begin{longtable}[c]{r l}
		\toprule
		Parâmetro 			& Significado \\
		\midrule
		$\rho_S$			&  Densidade eletrônica do \emph{shell} \\
		$\rho_C$			&  Densidade eletrônica do \emph{core}  \\
		$\rho_w$			&  Densidade eletrônica da água			\\
		$R_S$			& Raio do \emph{shell} 						\\
		$R_C$			& Raio do \emph{core}						\\
		$J_1$			&  Função de Bessel do primeiro tipo e de primeira ordem\\
		$C_4$			&  1,523	\\
		$C_5$			&  0,1477 	\\						
		\bottomrule
		\caption{Parâmetros para a equação \ref{eqn:Fcs}}
		\label{tab_ap:Fcs} 
	\end{longtable}
	
	Os termos $R_S$ e $R_C$ podem ser calculados pelas expressões \ref{eqn:Rs} e \ref{eqn:Rc}
	
	\begin{equation}
	R_C(\varepsilon\theta) = \sqrt{R_C^2\sin^2\theta + \varepsilon^2R_c^2\cos^2\theta}
	\label{eqn:Rc}
	\end{equation}
	
	\begin{equation}
	R_C = \sqrt{\frac{V_{\text{surf, apolar}}}{V_{\text{surf, total}}}}R_S
	\label{eqn:Rs}
	\end{equation}
	
	\noindent onde V é o volume molecular das regiões do surfactante.
	
	
	\chapter{Softwares}
		Neste apêndice serão descritos alguns dos métodos computacionais criados durante a execução deste doutorado. Todos os scripts foram escritos na linguagem Python. O aluno fortemente recomenda essa linguagem para outros que desejam tratar, visualizar e entender seus dados. Python possui uma sintaxe simples, mas poderosa, grande número de pacotes matemáticos e científicos de qualidade, e é totalmente gratuito. Em especial, a conjunção de \emph{Jupyter Notebooks} (extensão \texttt{ipynb}) com um \emph{kernel} de Python é uma ferramenta muito poderosa e conveniente.
		
		Um curso de Python com foco em tratamento de dados foi elaborado pelo aluno, e se encontra disponível em um repositório no Github\footnote{\href{https://github.com/KarlClinckspoor/CursoPython}{https://github.com/KarlClinckspoor/CursoPython}}. Em brevo, o curso possui a seguinte estrutura:
		
		\begin{enumerate}
			\item ``Hello world'', strings, obtendo ajuda
			\item Operações matemáticas, variáveis
			\item Estruturas de dados
			\item Condicionais e loops
			\item Instalando e carregando módulos
			\item Definindo funções
			\item Matemática computacional com \emph{numpy}
			\item Carregando e manipulando dados com \emph{pandas}
			\item Criando gráficos com \emph{pyplot}
			\item Tarefas avançadas
		\end{enumerate}
		

	\section{Descrição e uso do software de tratamento de curvas de fluxo}
	\section{Softwares miscelâneos para tratamento de dados}
\end{apendicesenv}

%\begin{anexosenv}
%
%% Imprime uma página indicando o início dos anexos
%\partanexos
%
%% ---
%\chapter{Morbi ultrices rutrum lorem.}
%% ---
%\lipsum[30]
%
%% ---
%\chapter{Cras non urna sed feugiat cum sociis natoque penatibus et magnis dis
%parturient montes nascetur ridiculus mus}
%% ---
%
%\lipsum[31]
%
%% ---
%\chapter{Fusce facilisis lacinia dui}
%% ---
%
%\lipsum[32]
%
%\end{anexosenv}
